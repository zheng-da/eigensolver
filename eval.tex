\section{Experimental Evaluation}

We evaluate the performance of the SSD-based FlashEigen on multiple real-world
billion-scale graphs including a web-page graph with 3.4 billion vertices.
We first evaluate the performance
of the two most computationally intensive computation in the eigensolvers:
sparse matrix dense matrix multiplication and dense matrix multiplication.
We demonstrate the effectiveness of the optimizations on the two operations
and compare the performance of our external-memory implementation with
multiple in-memory implementations: \textit{(i)} our in-memory implementations,
\textit{(ii)} MKL and \textit{(iii)} Trilinos. We then evaluate the overall
performance of FlashEigen and compare with the original Anasazi eigensolvers.
We further demonstrate the scalability
of FlashEigen on a web graph of 3.4 billion vertices and 129 billion edges.

We conduct all experiments on a non-uniform memory architecture machine with
four Intel Xeon E7-4860 processors, clocked at 2.6 GHz, and 1TB memory of
DDR3-1600. Each processor has 12 cores. The machine has three LSI SAS 9300-8e
host bus adapters (HBA) connected to a SuperMicro storage chassis, in which
24 OCZ Intrepid 3000 SSDs are installed. The 24 SSDs together are capable of
delivering 12 GB/s for read and 10 GB/s for write at maximum. The machine runs
Linux kernel v3.13.0. We use 48 threads for most of experiments by default.

\begin{table}
\begin{center}
\footnotesize
\begin{tabular}{|c|c|c|c|c|}
\hline
Graph datasets & \# Vertices & \# Edges & Directed \\
\hline
Twitter \cite{twitter} & $42$M & $1.5$B & Yes \\
\hline
Friendster \cite{friendster} & $65$M & $1.7$B & No \\
\hline
KNN distance graph \cite{} & $62$M & $12$B & No \\
\hline
Page \cite{web_graph} & $3.4$B & $129$B & Yes \\
%\hline
%RMAT graph \cite{rmat} & & & & \\
\hline
\end{tabular}
\normalsize
\end{center}
\vspace{-10pt}
\caption{Graph data sets.}
\label{graphs}
\end{table}

We use the real-world graphs in Table \ref{graphs} for evaluation. The largest
graph is the page graph with 3.4 billion vertices and 129 billion edges.
The smallest graph we use has 42 million vertices and 1.5 billion edges.
Twitter and Friendster are social network graphs. The KNN distance graph is
a symmetrized 100-nearest neighbor adjacency graph with cosine distance as
the edge weight. The distance graph is generated over all frames of the Babel
Tagalog corpus, commonly used in speech recognition. Majority of the vertices
in this graph has degree between $100$ to $1000$, so this graph does not follow
the power-law distribution in vertex degree.
The page graph is clustered by domain, generating good CPU cache hit rates
in sparse matrix dense matrix multiplication.

For performance evaluation, we implement an in-memory implementation of both
sparse matrix multiplication and dense matrix multiplication. Therefore, our
FlashEigen eigensolver is able to run both in memory and on SSDs.
In the following sections, we denote the eigensolver in memory by FE-IM and
the eigensolver on SSDs with FM-SEM.

\subsection{Sparse matrix multiplication}
This section evaluates the performance of the semi-external-memory (SEM)
implementation of SpMM in FlashEigen. We first evaluate the effectiveness
of the optimizations on the SpMM. Then we compare its performance with that
of the Intel MKL and Trilinos implementations.

\subsubsection{Optimizations on SpMM}
We first deploy a set of memory optimizations to implement a fast
in-memory sparse matrix multiplication and then deploy a set of I/O
optimizations to further speed up this operation in semi-external memory.
We apply the memory and I/O optimizations incrementally to illustrate their
effectiveness.

For memory optimizations, we start with an implementation that performs sparse
matrix multiplication on a sparse matrix in the CSR format and apply
the optimizations incrementally in the following order:
\begin{itemize} \itemsep1pt \parskip0pt \parsep0pt
		\item partition dense matrices for NUMA (\textit{NUMA}),
	\item partition the sparse matrix in both dimensions into tiles of
		$16K \times 16K$ (\textit{Cache blocking}),
	\item organize multiple physical tiles into a super tile to fill CPU cache
		(\textit{Super tile}),
	\item use CPU vectorization instructions (\textit{Vec}),
	\item allocate a local buffer to store the intermediate result of
		multiplication of tiles and the input dense matrix(\textit{Local write}),
	\item combine the SCSR and COO format to reduce the number of conditional
		jump CPU instructions (\textit{SCSR+COO}),
\end{itemize}

Figure \ref{perf:spmm_opt} shows that almost all of these optimizations have
positive effect on sparse matrix multiplication and all optimizations
together speed up the operation by $2-4$ times.
The degree of effectiveness varies
significantly between different graphs and different numbers of columns in
the dense matrices. For example, the NUMA optimization is more effective when
the dense matrices have more columns because more columns in the dense matrices
require more memory bandwidth. Cache blocking is very effective when
the dense matrices have fewer columns because it can effectively increase CPU
cache hits. When there are more columns in the dense matrices, data locality
improves and cache blocking becomes less effective. When there are too many
columns, the rows from
the input and output matrices can no longer be in the CPU cache. Thus, it even
has a little negative effect on the Friendster graph when the dense matrices
have $16$ columns. However, we never use dense matrices with more than four
columns in sparse matrix multiplication in the KrylovSchur eigensolver.
With all optimizations, we have a fast in-memory sparse matrix dense matrix
multiplication, denoted by FE-IM SpMM.

\begin{figure}
	\begin{center}
		\footnotesize
		\vspace{-15pt}
		\begin{tikzpicture}[gnuplot]
%% generated with GNUPLOT 4.6p4 (Lua 5.1; terminal rev. 99, script rev. 100)
%% Fri 25 Sep 2015 09:47:37 AM EDT
\path (0.000,0.000) rectangle (8.382,4.572);
\gpcolor{color=gp lt color border}
\gpsetlinetype{gp lt border}
\gpsetlinewidth{1.00}
\draw[gp path] (1.320,0.563)--(1.500,0.563);
\draw[gp path] (7.705,0.563)--(7.525,0.563);
\node[gp node right] at (1.136,0.563) { 0};
\draw[gp path] (1.320,0.796)--(1.500,0.796);
\draw[gp path] (7.705,0.796)--(7.525,0.796);
\node[gp node right] at (1.136,0.796) { 2};
\draw[gp path] (1.320,1.030)--(1.500,1.030);
\draw[gp path] (7.705,1.030)--(7.525,1.030);
\node[gp node right] at (1.136,1.030) { 4};
\draw[gp path] (1.320,1.263)--(1.500,1.263);
\draw[gp path] (7.705,1.263)--(7.525,1.263);
\node[gp node right] at (1.136,1.263) { 6};
\draw[gp path] (1.320,1.496)--(1.500,1.496);
\draw[gp path] (7.705,1.496)--(7.525,1.496);
\node[gp node right] at (1.136,1.496) { 8};
\draw[gp path] (1.320,1.730)--(1.500,1.730);
\draw[gp path] (7.705,1.730)--(7.525,1.730);
\node[gp node right] at (1.136,1.730) { 10};
\draw[gp path] (1.320,1.963)--(1.500,1.963);
\draw[gp path] (7.705,1.963)--(7.525,1.963);
\node[gp node right] at (1.136,1.963) { 12};
\draw[gp path] (1.320,2.196)--(1.500,2.196);
\draw[gp path] (7.705,2.196)--(7.525,2.196);
\node[gp node right] at (1.136,2.196) { 14};
\draw[gp path] (1.320,2.430)--(1.500,2.430);
\draw[gp path] (7.705,2.430)--(7.525,2.430);
\node[gp node right] at (1.136,2.430) { 16};
\draw[gp path] (1.320,2.663)--(1.500,2.663);
\draw[gp path] (7.705,2.663)--(7.525,2.663);
\node[gp node right] at (1.136,2.663) { 18};
\draw[gp path] (2.232,0.563)--(2.232,0.743);
\draw[gp path] (2.232,2.663)--(2.232,2.483);
\node[gp node left,rotate=-10] at (2.232,0.379) {F-1col};
\draw[gp path] (3.144,0.563)--(3.144,0.743);
\draw[gp path] (3.144,2.663)--(3.144,2.483);
\node[gp node left,rotate=-10] at (3.144,0.379) {F-4cols};
\draw[gp path] (4.056,0.563)--(4.056,0.743);
\draw[gp path] (4.056,2.663)--(4.056,2.483);
\node[gp node left,rotate=-10] at (4.056,0.379) {F-16cols};
\draw[gp path] (4.969,0.563)--(4.969,0.743);
\draw[gp path] (4.969,2.663)--(4.969,2.483);
\node[gp node left,rotate=-10] at (4.969,0.379) {T-1col};
\draw[gp path] (5.881,0.563)--(5.881,0.743);
\draw[gp path] (5.881,2.663)--(5.881,2.483);
\node[gp node left,rotate=-10] at (5.881,0.379) {T-4cols};
\draw[gp path] (6.793,0.563)--(6.793,0.743);
\draw[gp path] (6.793,2.663)--(6.793,2.483);
\node[gp node left,rotate=-10] at (6.793,0.379) {T-16cols};
\draw[gp path] (1.320,2.663)--(1.320,0.563)--(7.705,0.563)--(7.705,2.663)--cycle;
\node[gp node center,rotate=-270] at (0.246,1.613) {seconds};
\node[gp node right] at (3.228,4.238) {Base};
\def\gpfillpath{(3.412,4.161)--(4.328,4.161)--(4.328,4.315)--(3.412,4.315)--cycle}
\gpfill{color=gpbgfillcolor} \gpfillpath;
\gpfill{color=gp lt color 0,gp pattern 0,pattern color=.} \gpfillpath;
\gpcolor{color=gp lt color 0}
\gpsetlinetype{gp lt plot 0}
\draw[gp path] (3.412,4.161)--(4.328,4.161)--(4.328,4.315)--(3.412,4.315)--cycle;
\def\gpfillpath{(1.928,0.563)--(2.030,0.563)--(2.030,1.088)--(1.928,1.088)--cycle}
\gpfill{color=gpbgfillcolor} \gpfillpath;
\gpfill{color=gp lt color 0,gp pattern 0,pattern color=.} \gpfillpath;
\draw[gp path] (1.928,0.563)--(1.928,1.087)--(2.029,1.087)--(2.029,0.563)--cycle;
\def\gpfillpath{(2.840,0.563)--(2.943,0.563)--(2.943,1.550)--(2.840,1.550)--cycle}
\gpfill{color=gpbgfillcolor} \gpfillpath;
\gpfill{color=gp lt color 0,gp pattern 0,pattern color=.} \gpfillpath;
\draw[gp path] (2.840,0.563)--(2.840,1.549)--(2.942,1.549)--(2.942,0.563)--cycle;
\def\gpfillpath{(3.752,0.563)--(3.855,0.563)--(3.855,2.616)--(3.752,2.616)--cycle}
\gpfill{color=gpbgfillcolor} \gpfillpath;
\gpfill{color=gp lt color 0,gp pattern 0,pattern color=.} \gpfillpath;
\draw[gp path] (3.752,0.563)--(3.752,2.615)--(3.854,2.615)--(3.854,0.563)--cycle;
\def\gpfillpath{(4.665,0.563)--(4.767,0.563)--(4.767,0.857)--(4.665,0.857)--cycle}
\gpfill{color=gpbgfillcolor} \gpfillpath;
\gpfill{color=gp lt color 0,gp pattern 0,pattern color=.} \gpfillpath;
\draw[gp path] (4.665,0.563)--(4.665,0.856)--(4.766,0.856)--(4.766,0.563)--cycle;
\def\gpfillpath{(5.577,0.563)--(5.679,0.563)--(5.679,1.052)--(5.577,1.052)--cycle}
\gpfill{color=gpbgfillcolor} \gpfillpath;
\gpfill{color=gp lt color 0,gp pattern 0,pattern color=.} \gpfillpath;
\draw[gp path] (5.577,0.563)--(5.577,1.051)--(5.678,1.051)--(5.678,0.563)--cycle;
\def\gpfillpath{(6.489,0.563)--(6.591,0.563)--(6.591,1.603)--(6.489,1.603)--cycle}
\gpfill{color=gpbgfillcolor} \gpfillpath;
\gpfill{color=gp lt color 0,gp pattern 0,pattern color=.} \gpfillpath;
\draw[gp path] (6.489,0.563)--(6.489,1.602)--(6.590,1.602)--(6.590,0.563)--cycle;
\gpcolor{color=gp lt color border}
\node[gp node right] at (3.228,3.930) {NUMA};
\def\gpfillpath{(3.412,3.853)--(4.328,3.853)--(4.328,4.007)--(3.412,4.007)--cycle}
\gpfill{color=gpbgfillcolor} \gpfillpath;
\gpfill{color=gp lt color 1,gp pattern 1,pattern color=.} \gpfillpath;
\gpcolor{color=gp lt color 1}
\gpsetlinetype{gp lt plot 1}
\draw[gp path] (3.412,3.853)--(4.328,3.853)--(4.328,4.007)--(3.412,4.007)--cycle;
\def\gpfillpath{(2.029,0.563)--(2.132,0.563)--(2.132,1.024)--(2.029,1.024)--cycle}
\gpfill{color=gpbgfillcolor} \gpfillpath;
\gpfill{color=gp lt color 1,gp pattern 1,pattern color=.} \gpfillpath;
\draw[gp path] (2.029,0.563)--(2.029,1.023)--(2.131,1.023)--(2.131,0.563)--cycle;
\def\gpfillpath{(2.942,0.563)--(3.044,0.563)--(3.044,1.230)--(2.942,1.230)--cycle}
\gpfill{color=gpbgfillcolor} \gpfillpath;
\gpfill{color=gp lt color 1,gp pattern 1,pattern color=.} \gpfillpath;
\draw[gp path] (2.942,0.563)--(2.942,1.229)--(3.043,1.229)--(3.043,0.563)--cycle;
\def\gpfillpath{(3.854,0.563)--(3.956,0.563)--(3.956,1.726)--(3.854,1.726)--cycle}
\gpfill{color=gpbgfillcolor} \gpfillpath;
\gpfill{color=gp lt color 1,gp pattern 1,pattern color=.} \gpfillpath;
\draw[gp path] (3.854,0.563)--(3.854,1.725)--(3.955,1.725)--(3.955,0.563)--cycle;
\def\gpfillpath{(4.766,0.563)--(4.868,0.563)--(4.868,0.816)--(4.766,0.816)--cycle}
\gpfill{color=gpbgfillcolor} \gpfillpath;
\gpfill{color=gp lt color 1,gp pattern 1,pattern color=.} \gpfillpath;
\draw[gp path] (4.766,0.563)--(4.766,0.815)--(4.867,0.815)--(4.867,0.563)--cycle;
\def\gpfillpath{(5.678,0.563)--(5.780,0.563)--(5.780,0.940)--(5.678,0.940)--cycle}
\gpfill{color=gpbgfillcolor} \gpfillpath;
\gpfill{color=gp lt color 1,gp pattern 1,pattern color=.} \gpfillpath;
\draw[gp path] (5.678,0.563)--(5.678,0.939)--(5.779,0.939)--(5.779,0.563)--cycle;
\def\gpfillpath{(6.590,0.563)--(6.693,0.563)--(6.693,1.236)--(6.590,1.236)--cycle}
\gpfill{color=gpbgfillcolor} \gpfillpath;
\gpfill{color=gp lt color 1,gp pattern 1,pattern color=.} \gpfillpath;
\draw[gp path] (6.590,0.563)--(6.590,1.235)--(6.692,1.235)--(6.692,0.563)--cycle;
\gpcolor{color=gp lt color border}
\node[gp node right] at (3.228,3.622) {Cache blocking};
\def\gpfillpath{(3.412,3.545)--(4.328,3.545)--(4.328,3.699)--(3.412,3.699)--cycle}
\gpfill{color=gpbgfillcolor} \gpfillpath;
\gpfill{color=gp lt color 2,gp pattern 2,pattern color=.} \gpfillpath;
\gpcolor{color=gp lt color 2}
\gpsetlinetype{gp lt plot 2}
\draw[gp path] (3.412,3.545)--(4.328,3.545)--(4.328,3.699)--(3.412,3.699)--cycle;
\def\gpfillpath{(2.131,0.563)--(2.233,0.563)--(2.233,0.893)--(2.131,0.893)--cycle}
\gpfill{color=gpbgfillcolor} \gpfillpath;
\gpfill{color=gp lt color 2,gp pattern 2,pattern color=.} \gpfillpath;
\draw[gp path] (2.131,0.563)--(2.131,0.892)--(2.232,0.892)--(2.232,0.563)--cycle;
\def\gpfillpath{(3.043,0.563)--(3.145,0.563)--(3.145,1.182)--(3.043,1.182)--cycle}
\gpfill{color=gpbgfillcolor} \gpfillpath;
\gpfill{color=gp lt color 2,gp pattern 2,pattern color=.} \gpfillpath;
\draw[gp path] (3.043,0.563)--(3.043,1.181)--(3.144,1.181)--(3.144,0.563)--cycle;
\def\gpfillpath{(3.955,0.563)--(4.057,0.563)--(4.057,1.851)--(3.955,1.851)--cycle}
\gpfill{color=gpbgfillcolor} \gpfillpath;
\gpfill{color=gp lt color 2,gp pattern 2,pattern color=.} \gpfillpath;
\draw[gp path] (3.955,0.563)--(3.955,1.850)--(4.056,1.850)--(4.056,0.563)--cycle;
\def\gpfillpath{(4.867,0.563)--(4.970,0.563)--(4.970,0.659)--(4.867,0.659)--cycle}
\gpfill{color=gpbgfillcolor} \gpfillpath;
\gpfill{color=gp lt color 2,gp pattern 2,pattern color=.} \gpfillpath;
\draw[gp path] (4.867,0.563)--(4.867,0.658)--(4.969,0.658)--(4.969,0.563)--cycle;
\def\gpfillpath{(5.779,0.563)--(5.882,0.563)--(5.882,0.766)--(5.779,0.766)--cycle}
\gpfill{color=gpbgfillcolor} \gpfillpath;
\gpfill{color=gp lt color 2,gp pattern 2,pattern color=.} \gpfillpath;
\draw[gp path] (5.779,0.563)--(5.779,0.765)--(5.881,0.765)--(5.881,0.563)--cycle;
\def\gpfillpath{(6.692,0.563)--(6.794,0.563)--(6.794,0.994)--(6.692,0.994)--cycle}
\gpfill{color=gpbgfillcolor} \gpfillpath;
\gpfill{color=gp lt color 2,gp pattern 2,pattern color=.} \gpfillpath;
\draw[gp path] (6.692,0.563)--(6.692,0.993)--(6.793,0.993)--(6.793,0.563)--cycle;
\gpcolor{color=gp lt color border}
\node[gp node right] at (3.228,3.314) {Super tile};
\def\gpfillpath{(3.412,3.237)--(4.328,3.237)--(4.328,3.391)--(3.412,3.391)--cycle}
\gpfill{color=gpbgfillcolor} \gpfillpath;
\gpfill{color=gp lt color 3,gp pattern 3,pattern color=.} \gpfillpath;
\gpcolor{color=gp lt color 3}
\gpsetlinetype{gp lt plot 3}
\draw[gp path] (3.412,3.237)--(4.328,3.237)--(4.328,3.391)--(3.412,3.391)--cycle;
\def\gpfillpath{(2.232,0.563)--(2.334,0.563)--(2.334,0.856)--(2.232,0.856)--cycle}
\gpfill{color=gpbgfillcolor} \gpfillpath;
\gpfill{color=gp lt color 3,gp pattern 3,pattern color=.} \gpfillpath;
\draw[gp path] (2.232,0.563)--(2.232,0.855)--(2.333,0.855)--(2.333,0.563)--cycle;
\def\gpfillpath{(3.144,0.563)--(3.247,0.563)--(3.247,1.194)--(3.144,1.194)--cycle}
\gpfill{color=gpbgfillcolor} \gpfillpath;
\gpfill{color=gp lt color 3,gp pattern 3,pattern color=.} \gpfillpath;
\draw[gp path] (3.144,0.563)--(3.144,1.193)--(3.246,1.193)--(3.246,0.563)--cycle;
\def\gpfillpath{(4.056,0.563)--(4.159,0.563)--(4.159,1.845)--(4.056,1.845)--cycle}
\gpfill{color=gpbgfillcolor} \gpfillpath;
\gpfill{color=gp lt color 3,gp pattern 3,pattern color=.} \gpfillpath;
\draw[gp path] (4.056,0.563)--(4.056,1.844)--(4.158,1.844)--(4.158,0.563)--cycle;
\def\gpfillpath{(4.969,0.563)--(5.071,0.563)--(5.071,0.656)--(4.969,0.656)--cycle}
\gpfill{color=gpbgfillcolor} \gpfillpath;
\gpfill{color=gp lt color 3,gp pattern 3,pattern color=.} \gpfillpath;
\draw[gp path] (4.969,0.563)--(4.969,0.655)--(5.070,0.655)--(5.070,0.563)--cycle;
\def\gpfillpath{(5.881,0.563)--(5.983,0.563)--(5.983,0.765)--(5.881,0.765)--cycle}
\gpfill{color=gpbgfillcolor} \gpfillpath;
\gpfill{color=gp lt color 3,gp pattern 3,pattern color=.} \gpfillpath;
\draw[gp path] (5.881,0.563)--(5.881,0.764)--(5.982,0.764)--(5.982,0.563)--cycle;
\def\gpfillpath{(6.793,0.563)--(6.895,0.563)--(6.895,0.989)--(6.793,0.989)--cycle}
\gpfill{color=gpbgfillcolor} \gpfillpath;
\gpfill{color=gp lt color 3,gp pattern 3,pattern color=.} \gpfillpath;
\draw[gp path] (6.793,0.563)--(6.793,0.988)--(6.894,0.988)--(6.894,0.563)--cycle;
\gpcolor{color=gp lt color border}
\node[gp node right] at (7.088,4.238) {Vec};
\def\gpfillpath{(7.272,4.161)--(8.188,4.161)--(8.188,4.315)--(7.272,4.315)--cycle}
\gpfill{color=gpbgfillcolor} \gpfillpath;
\gpfill{color=gp lt color 4,gp pattern 4,pattern color=.} \gpfillpath;
\gpcolor{color=gp lt color 4}
\gpsetlinetype{gp lt plot 4}
\draw[gp path] (7.272,4.161)--(8.188,4.161)--(8.188,4.315)--(7.272,4.315)--cycle;
\def\gpfillpath{(2.333,0.563)--(2.436,0.563)--(2.436,0.804)--(2.333,0.804)--cycle}
\gpfill{color=gpbgfillcolor} \gpfillpath;
\gpfill{color=gp lt color 4,gp pattern 4,pattern color=.} \gpfillpath;
\draw[gp path] (2.333,0.563)--(2.333,0.803)--(2.435,0.803)--(2.435,0.563)--cycle;
\def\gpfillpath{(3.246,0.563)--(3.348,0.563)--(3.348,1.006)--(3.246,1.006)--cycle}
\gpfill{color=gpbgfillcolor} \gpfillpath;
\gpfill{color=gp lt color 4,gp pattern 4,pattern color=.} \gpfillpath;
\draw[gp path] (3.246,0.563)--(3.246,1.005)--(3.347,1.005)--(3.347,0.563)--cycle;
\def\gpfillpath{(4.158,0.563)--(4.260,0.563)--(4.260,1.653)--(4.158,1.653)--cycle}
\gpfill{color=gpbgfillcolor} \gpfillpath;
\gpfill{color=gp lt color 4,gp pattern 4,pattern color=.} \gpfillpath;
\draw[gp path] (4.158,0.563)--(4.158,1.652)--(4.259,1.652)--(4.259,0.563)--cycle;
\def\gpfillpath{(5.070,0.563)--(5.172,0.563)--(5.172,0.644)--(5.070,0.644)--cycle}
\gpfill{color=gpbgfillcolor} \gpfillpath;
\gpfill{color=gp lt color 4,gp pattern 4,pattern color=.} \gpfillpath;
\draw[gp path] (5.070,0.563)--(5.070,0.643)--(5.171,0.643)--(5.171,0.563)--cycle;
\def\gpfillpath{(5.982,0.563)--(6.084,0.563)--(6.084,0.709)--(5.982,0.709)--cycle}
\gpfill{color=gpbgfillcolor} \gpfillpath;
\gpfill{color=gp lt color 4,gp pattern 4,pattern color=.} \gpfillpath;
\draw[gp path] (5.982,0.563)--(5.982,0.708)--(6.083,0.708)--(6.083,0.563)--cycle;
\def\gpfillpath{(6.894,0.563)--(6.997,0.563)--(6.997,0.919)--(6.894,0.919)--cycle}
\gpfill{color=gpbgfillcolor} \gpfillpath;
\gpfill{color=gp lt color 4,gp pattern 4,pattern color=.} \gpfillpath;
\draw[gp path] (6.894,0.563)--(6.894,0.918)--(6.996,0.918)--(6.996,0.563)--cycle;
\gpcolor{color=gp lt color border}
\node[gp node right] at (7.088,3.930) {Local write};
\def\gpfillpath{(7.272,3.853)--(8.188,3.853)--(8.188,4.007)--(7.272,4.007)--cycle}
\gpfill{color=gpbgfillcolor} \gpfillpath;
\gpfill{color=gp lt color 5,gp pattern 5,pattern color=.} \gpfillpath;
\gpcolor{color=gp lt color 5}
\gpsetlinetype{gp lt plot 5}
\draw[gp path] (7.272,3.853)--(8.188,3.853)--(8.188,4.007)--(7.272,4.007)--cycle;
\def\gpfillpath{(2.435,0.563)--(2.537,0.563)--(2.537,0.792)--(2.435,0.792)--cycle}
\gpfill{color=gpbgfillcolor} \gpfillpath;
\gpfill{color=gp lt color 5,gp pattern 5,pattern color=.} \gpfillpath;
\draw[gp path] (2.435,0.563)--(2.435,0.791)--(2.536,0.791)--(2.536,0.563)--cycle;
\def\gpfillpath{(3.347,0.563)--(3.449,0.563)--(3.449,0.972)--(3.347,0.972)--cycle}
\gpfill{color=gpbgfillcolor} \gpfillpath;
\gpfill{color=gp lt color 5,gp pattern 5,pattern color=.} \gpfillpath;
\draw[gp path] (3.347,0.563)--(3.347,0.971)--(3.448,0.971)--(3.448,0.563)--cycle;
\def\gpfillpath{(4.259,0.563)--(4.361,0.563)--(4.361,1.599)--(4.259,1.599)--cycle}
\gpfill{color=gpbgfillcolor} \gpfillpath;
\gpfill{color=gp lt color 5,gp pattern 5,pattern color=.} \gpfillpath;
\draw[gp path] (4.259,0.563)--(4.259,1.598)--(4.360,1.598)--(4.360,0.563)--cycle;
\def\gpfillpath{(5.171,0.563)--(5.274,0.563)--(5.274,0.639)--(5.171,0.639)--cycle}
\gpfill{color=gpbgfillcolor} \gpfillpath;
\gpfill{color=gp lt color 5,gp pattern 5,pattern color=.} \gpfillpath;
\draw[gp path] (5.171,0.563)--(5.171,0.638)--(5.273,0.638)--(5.273,0.563)--cycle;
\def\gpfillpath{(6.083,0.563)--(6.186,0.563)--(6.186,0.702)--(6.083,0.702)--cycle}
\gpfill{color=gpbgfillcolor} \gpfillpath;
\gpfill{color=gp lt color 5,gp pattern 5,pattern color=.} \gpfillpath;
\draw[gp path] (6.083,0.563)--(6.083,0.701)--(6.185,0.701)--(6.185,0.563)--cycle;
\def\gpfillpath{(6.996,0.563)--(7.098,0.563)--(7.098,0.893)--(6.996,0.893)--cycle}
\gpfill{color=gpbgfillcolor} \gpfillpath;
\gpfill{color=gp lt color 5,gp pattern 5,pattern color=.} \gpfillpath;
\draw[gp path] (6.996,0.563)--(6.996,0.892)--(7.097,0.892)--(7.097,0.563)--cycle;
\gpcolor{color=gp lt color border}
\node[gp node right] at (7.088,3.622) {SCSR+COO};
\def\gpfillpath{(7.272,3.545)--(8.188,3.545)--(8.188,3.699)--(7.272,3.699)--cycle}
\gpfill{color=gpbgfillcolor} \gpfillpath;
\gpfill{color=gp lt color 6,gp pattern 6,pattern color=.} \gpfillpath;
\gpcolor{color=gp lt color 6}
\gpsetlinetype{gp lt plot 6}
\draw[gp path] (7.272,3.545)--(8.188,3.545)--(8.188,3.699)--(7.272,3.699)--cycle;
\def\gpfillpath{(2.536,0.563)--(2.639,0.563)--(2.639,0.765)--(2.536,0.765)--cycle}
\gpfill{color=gpbgfillcolor} \gpfillpath;
\gpfill{color=gp lt color 6,gp pattern 6,pattern color=.} \gpfillpath;
\draw[gp path] (2.536,0.563)--(2.536,0.764)--(2.638,0.764)--(2.638,0.563)--cycle;
\def\gpfillpath{(3.448,0.563)--(3.551,0.563)--(3.551,0.943)--(3.448,0.943)--cycle}
\gpfill{color=gpbgfillcolor} \gpfillpath;
\gpfill{color=gp lt color 6,gp pattern 6,pattern color=.} \gpfillpath;
\draw[gp path] (3.448,0.563)--(3.448,0.942)--(3.550,0.942)--(3.550,0.563)--cycle;
\def\gpfillpath{(4.360,0.563)--(4.463,0.563)--(4.463,1.553)--(4.360,1.553)--cycle}
\gpfill{color=gpbgfillcolor} \gpfillpath;
\gpfill{color=gp lt color 6,gp pattern 6,pattern color=.} \gpfillpath;
\draw[gp path] (4.360,0.563)--(4.360,1.552)--(4.462,1.552)--(4.462,0.563)--cycle;
\def\gpfillpath{(5.273,0.563)--(5.375,0.563)--(5.375,0.637)--(5.273,0.637)--cycle}
\gpfill{color=gpbgfillcolor} \gpfillpath;
\gpfill{color=gp lt color 6,gp pattern 6,pattern color=.} \gpfillpath;
\draw[gp path] (5.273,0.563)--(5.273,0.636)--(5.374,0.636)--(5.374,0.563)--cycle;
\def\gpfillpath{(6.185,0.563)--(6.287,0.563)--(6.287,0.699)--(6.185,0.699)--cycle}
\gpfill{color=gpbgfillcolor} \gpfillpath;
\gpfill{color=gp lt color 6,gp pattern 6,pattern color=.} \gpfillpath;
\draw[gp path] (6.185,0.563)--(6.185,0.698)--(6.286,0.698)--(6.286,0.563)--cycle;
\def\gpfillpath{(7.097,0.563)--(7.199,0.563)--(7.199,0.882)--(7.097,0.882)--cycle}
\gpfill{color=gpbgfillcolor} \gpfillpath;
\gpfill{color=gp lt color 6,gp pattern 6,pattern color=.} \gpfillpath;
\draw[gp path] (7.097,0.563)--(7.097,0.881)--(7.198,0.881)--(7.198,0.563)--cycle;
\gpcolor{color=gp lt color border}
\gpsetlinetype{gp lt border}
\draw[gp path] (1.320,2.663)--(1.320,0.563)--(7.705,0.563)--(7.705,2.663)--cycle;
%% coordinates of the plot area
\gpdefrectangularnode{gp plot 1}{\pgfpoint{1.320cm}{0.563cm}}{\pgfpoint{7.705cm}{2.663cm}}
\end{tikzpicture}
%% gnuplot variables

		\vspace{-15pt}
		\caption{The effectiveness of the SpMM optimizations on the Friendster
			graph (F) and the Twitter graph (T) for different numbers of
			columns in the dense matrices.}
		\label{perf:spmm_opt}
	\end{center}
\end{figure}

\subsubsection{SpMM performance}

We then evaluate the performance of the SEM SpMM in FlashEigen and compare its
performance with FE-IM SpMM, the MKL implementation and the Trilinos
implementation (Figure \ref{perf:spmm}). We only show the performance of
SpMM on the Friendster graph and the performance on the other graphs is
similar. The MKL and Trilinos SpMM cannot run on the page graph.

Our SEM SpMM has performance comparable to FE-IM SpMM, while both FE-IM SpMM
and FE-SEM SpMM outperforms the MKL and Trilinos implementations.
On the Friendster graph, FE-SEM SpMM achieves 60\% performance of FE-IM SpMM
when the dense matrix has only one column and the performance gap narrows
as the number of columns in the dense matrices increases.
The Trilinos SpMM is optimized for sparse matrix vector multiplication (SpMV).
But even for SpMV, our IM-SpMM outperforms Trilinos by 36\%. The MKL SpMM
performs better when the dense matrices have more columns, but our
implementations can still outperform MKL by $2-3$ times in most settings.

\begin{figure}
	\begin{center}
		\footnotesize
		\vspace{-15pt}
		\begin{tikzpicture}[gnuplot]
%% generated with GNUPLOT 4.6p4 (Lua 5.1; terminal rev. 99, script rev. 100)
%% Tue 06 Oct 2015 10:20:02 PM EDT
\path (0.000,0.000) rectangle (8.382,4.572);
\gpcolor{color=gp lt color border}
\gpsetlinetype{gp lt border}
\gpsetlinewidth{1.00}
\draw[gp path] (1.320,0.985)--(1.500,0.985);
\draw[gp path] (7.829,0.985)--(7.649,0.985);
\node[gp node right] at (1.136,0.985) { 0};
\draw[gp path] (1.320,1.313)--(1.500,1.313);
\draw[gp path] (7.829,1.313)--(7.649,1.313);
\node[gp node right] at (1.136,1.313) { 5};
\draw[gp path] (1.320,1.640)--(1.500,1.640);
\draw[gp path] (7.829,1.640)--(7.649,1.640);
\node[gp node right] at (1.136,1.640) { 10};
\draw[gp path] (1.320,1.968)--(1.500,1.968);
\draw[gp path] (7.829,1.968)--(7.649,1.968);
\node[gp node right] at (1.136,1.968) { 15};
\draw[gp path] (1.320,2.296)--(1.500,2.296);
\draw[gp path] (7.829,2.296)--(7.649,2.296);
\node[gp node right] at (1.136,2.296) { 20};
\draw[gp path] (1.320,2.624)--(1.500,2.624);
\draw[gp path] (7.829,2.624)--(7.649,2.624);
\node[gp node right] at (1.136,2.624) { 25};
\draw[gp path] (1.320,2.951)--(1.500,2.951);
\draw[gp path] (7.829,2.951)--(7.649,2.951);
\node[gp node right] at (1.136,2.951) { 30};
\draw[gp path] (1.320,3.279)--(1.500,3.279);
\draw[gp path] (7.829,3.279)--(7.649,3.279);
\node[gp node right] at (1.136,3.279) { 35};
\draw[gp path] (1.320,0.985)--(1.320,1.165);
\draw[gp path] (1.320,3.279)--(1.320,3.099);
\node[gp node center] at (1.320,0.677) {1};
\draw[gp path] (2.947,0.985)--(2.947,1.165);
\draw[gp path] (2.947,3.279)--(2.947,3.099);
\node[gp node center] at (2.947,0.677) {2};
\draw[gp path] (4.575,0.985)--(4.575,1.165);
\draw[gp path] (4.575,3.279)--(4.575,3.099);
\node[gp node center] at (4.575,0.677) {4};
\draw[gp path] (6.202,0.985)--(6.202,1.165);
\draw[gp path] (6.202,3.279)--(6.202,3.099);
\node[gp node center] at (6.202,0.677) {8};
\draw[gp path] (7.829,0.985)--(7.829,1.165);
\draw[gp path] (7.829,3.279)--(7.829,3.099);
\node[gp node center] at (7.829,0.677) {16};
\draw[gp path] (1.320,3.279)--(1.320,0.985)--(7.829,0.985)--(7.829,3.279)--cycle;
\node[gp node center,rotate=-270] at (0.246,2.132) {seconds};
\node[gp node center] at (4.574,0.215) {The number of columns in the dense matrix};
\node[gp node right] at (3.290,4.238) {FE-IM};
\gpcolor{color=gp lt color 0}
\gpsetlinetype{gp lt plot 0}
\draw[gp path] (3.474,4.238)--(4.390,4.238);
\draw[gp path] (1.320,1.094)--(2.947,1.127)--(4.575,1.199)--(6.202,1.311)--(7.829,1.573);
\gpsetpointsize{4.00}
\gppoint{gp mark 1}{(1.320,1.094)}
\gppoint{gp mark 1}{(2.947,1.127)}
\gppoint{gp mark 1}{(4.575,1.199)}
\gppoint{gp mark 1}{(6.202,1.311)}
\gppoint{gp mark 1}{(7.829,1.573)}
\gppoint{gp mark 1}{(3.932,4.238)}
\gpcolor{color=gp lt color border}
\node[gp node right] at (3.290,3.930) {FE-SEM};
\gpcolor{color=gp lt color 1}
\gpsetlinetype{gp lt plot 1}
\draw[gp path] (3.474,3.930)--(4.390,3.930);
\draw[gp path] (1.320,1.150)--(2.947,1.158)--(4.575,1.242)--(6.202,1.323)--(7.829,1.628);
\gppoint{gp mark 2}{(1.320,1.150)}
\gppoint{gp mark 2}{(2.947,1.158)}
\gppoint{gp mark 2}{(4.575,1.242)}
\gppoint{gp mark 2}{(6.202,1.323)}
\gppoint{gp mark 2}{(7.829,1.628)}
\gppoint{gp mark 2}{(3.932,3.930)}
\gpcolor{color=gp lt color border}
\node[gp node right] at (6.046,4.238) {MKL};
\gpcolor{color=gp lt color 2}
\gpsetlinetype{gp lt plot 2}
\draw[gp path] (6.230,4.238)--(7.146,4.238);
\draw[gp path] (1.320,1.384)--(2.947,1.389)--(4.575,1.570)--(6.202,1.689)--(7.829,1.973);
\gppoint{gp mark 3}{(1.320,1.384)}
\gppoint{gp mark 3}{(2.947,1.389)}
\gppoint{gp mark 3}{(4.575,1.570)}
\gppoint{gp mark 3}{(6.202,1.689)}
\gppoint{gp mark 3}{(7.829,1.973)}
\gppoint{gp mark 3}{(6.688,4.238)}
\gpcolor{color=gp lt color border}
\node[gp node right] at (6.046,3.930) {Trilinos};
\gpcolor{color=gp lt color 3}
\gpsetlinetype{gp lt plot 3}
\draw[gp path] (6.230,3.930)--(7.146,3.930);
\draw[gp path] (1.320,1.155)--(2.947,1.281)--(4.575,1.530)--(6.202,2.036)--(7.829,3.126);
\gppoint{gp mark 4}{(1.320,1.155)}
\gppoint{gp mark 4}{(2.947,1.281)}
\gppoint{gp mark 4}{(4.575,1.530)}
\gppoint{gp mark 4}{(6.202,2.036)}
\gppoint{gp mark 4}{(7.829,3.126)}
\gppoint{gp mark 4}{(6.688,3.930)}
\gpcolor{color=gp lt color border}
\gpsetlinetype{gp lt border}
\draw[gp path] (1.320,3.279)--(1.320,0.985)--(7.829,0.985)--(7.829,3.279)--cycle;
%% coordinates of the plot area
\gpdefrectangularnode{gp plot 1}{\pgfpoint{1.320cm}{0.985cm}}{\pgfpoint{7.829cm}{3.279cm}}
\end{tikzpicture}
%% gnuplot variables

		\vspace{-15pt}
		\caption{The runtime of in-memory SpMM (FE-IM) and SEM-SpMM (FE-SEM)
			in the FlashEigen, the MKL and the Trilinos implementation on
		the Friendster graph.}
		\label{perf:spmm}
	\end{center}
\end{figure}

\begin{figure}[t]
\centering
\footnotesize
\vspace{-15pt}
\begin{subfigure}{.5\linewidth}
	\begin{tikzpicture}[gnuplot]
%% generated with GNUPLOT 4.6p4 (Lua 5.1; terminal rev. 99, script rev. 100)
%% Wed 11 Nov 2015 10:11:41 AM EST
\path (0.000,0.000) rectangle (5.588,4.572);
\gpcolor{color=gp lt color border}
\gpsetlinetype{gp lt border}
\gpsetlinewidth{1.00}
\draw[gp path] (1.196,1.126)--(1.376,1.126);
\draw[gp path] (4.345,1.126)--(4.165,1.126);
\node[gp node right] at (1.012,1.126) { 0};
\draw[gp path] (1.196,1.557)--(1.376,1.557);
\draw[gp path] (4.345,1.557)--(4.165,1.557);
\node[gp node right] at (1.012,1.557) { 0.2};
\draw[gp path] (1.196,1.987)--(1.376,1.987);
\draw[gp path] (4.345,1.987)--(4.165,1.987);
\node[gp node right] at (1.012,1.987) { 0.4};
\draw[gp path] (1.196,2.418)--(1.376,2.418);
\draw[gp path] (4.345,2.418)--(4.165,2.418);
\node[gp node right] at (1.012,2.418) { 0.6};
\draw[gp path] (1.196,2.848)--(1.376,2.848);
\draw[gp path] (4.345,2.848)--(4.165,2.848);
\node[gp node right] at (1.012,2.848) { 0.8};
\draw[gp path] (1.196,3.279)--(1.376,3.279);
\draw[gp path] (4.345,3.279)--(4.165,3.279);
\node[gp node right] at (1.012,3.279) { 1};
\draw[gp path] (1.721,1.126)--(1.721,1.306);
\draw[gp path] (1.721,3.279)--(1.721,3.099);
\node[gp node left,rotate=-20] at (1.721,0.942) {rmat-100M-40};
\draw[gp path] (2.246,1.126)--(2.246,1.306);
\draw[gp path] (2.246,3.279)--(2.246,3.099);
\node[gp node left,rotate=-20] at (2.246,0.942) {rmat-100M-160};
\draw[gp path] (2.771,1.126)--(2.771,1.306);
\draw[gp path] (2.771,3.279)--(2.771,3.099);
\node[gp node left,rotate=-20] at (2.771,0.942) {W};
\draw[gp path] (3.295,1.126)--(3.295,1.306);
\draw[gp path] (3.295,3.279)--(3.295,3.099);
\node[gp node left,rotate=-20] at (3.295,0.942) {Friendster};
\draw[gp path] (3.820,1.126)--(3.820,1.306);
\draw[gp path] (3.820,3.279)--(3.820,3.099);
\node[gp node left,rotate=-20] at (3.820,0.942) {Page graph};
\draw[gp path] (1.196,3.279)--(1.196,1.126)--(4.345,1.126)--(4.345,3.279)--cycle;
\node[gp node right] at (2.864,4.238) {Trilinos};
\def\gpfillpath{(3.048,4.161)--(3.964,4.161)--(3.964,4.315)--(3.048,4.315)--cycle}
\gpfill{color=gpbgfillcolor} \gpfillpath;
\gpfill{color=gp lt color 0,gp pattern 0,pattern color=.} \gpfillpath;
\gpcolor{color=gp lt color 0}
\gpsetlinetype{gp lt plot 0}
\draw[gp path] (3.048,4.161)--(3.964,4.161)--(3.964,4.315)--(3.048,4.315)--cycle;
\def\gpfillpath{(1.655,1.126)--(1.787,1.126)--(1.787,2.526)--(1.655,2.526)--cycle}
\gpfill{color=gpbgfillcolor} \gpfillpath;
\gpfill{color=gp lt color 0,gp pattern 0,pattern color=.} \gpfillpath;
\draw[gp path] (1.655,1.126)--(1.655,2.525)--(1.786,2.525)--(1.786,1.126)--cycle;
\def\gpfillpath{(2.180,1.126)--(2.312,1.126)--(2.312,2.182)--(2.180,2.182)--cycle}
\gpfill{color=gpbgfillcolor} \gpfillpath;
\gpfill{color=gp lt color 0,gp pattern 0,pattern color=.} \gpfillpath;
\draw[gp path] (2.180,1.126)--(2.180,2.181)--(2.311,2.181)--(2.311,1.126)--cycle;
\def\gpfillpath{(2.705,1.126)--(2.837,1.126)--(2.837,2.419)--(2.705,2.419)--cycle}
\gpfill{color=gpbgfillcolor} \gpfillpath;
\gpfill{color=gp lt color 0,gp pattern 0,pattern color=.} \gpfillpath;
\draw[gp path] (2.705,1.126)--(2.705,2.418)--(2.836,2.418)--(2.836,1.126)--cycle;
\def\gpfillpath{(3.230,1.126)--(3.362,1.126)--(3.362,2.397)--(3.230,2.397)--cycle}
\gpfill{color=gpbgfillcolor} \gpfillpath;
\gpfill{color=gp lt color 0,gp pattern 0,pattern color=.} \gpfillpath;
\draw[gp path] (3.230,1.126)--(3.230,2.396)--(3.361,2.396)--(3.361,1.126)--cycle;
\gpcolor{color=gp lt color border}
\node[gp node right] at (2.864,3.930) {SEM};
\def\gpfillpath{(3.048,3.853)--(3.964,3.853)--(3.964,4.007)--(3.048,4.007)--cycle}
\gpfill{color=gpbgfillcolor} \gpfillpath;
\gpfill{color=gp lt color 1,gp pattern 1,pattern color=.} \gpfillpath;
\gpcolor{color=gp lt color 1}
\gpsetlinetype{gp lt plot 1}
\draw[gp path] (3.048,3.853)--(3.964,3.853)--(3.964,4.007)--(3.048,4.007)--cycle;
\def\gpfillpath{(1.786,1.126)--(1.919,1.126)--(1.919,2.957)--(1.786,2.957)--cycle}
\gpfill{color=gpbgfillcolor} \gpfillpath;
\gpfill{color=gp lt color 1,gp pattern 1,pattern color=.} \gpfillpath;
\draw[gp path] (1.786,1.126)--(1.786,2.956)--(1.918,2.956)--(1.918,1.126)--cycle;
\def\gpfillpath{(2.311,1.126)--(2.443,1.126)--(2.443,2.634)--(2.311,2.634)--cycle}
\gpfill{color=gpbgfillcolor} \gpfillpath;
\gpfill{color=gp lt color 1,gp pattern 1,pattern color=.} \gpfillpath;
\draw[gp path] (2.311,1.126)--(2.311,2.633)--(2.442,2.633)--(2.442,1.126)--cycle;
\def\gpfillpath{(2.836,1.126)--(2.968,1.126)--(2.968,2.204)--(2.836,2.204)--cycle}
\gpfill{color=gpbgfillcolor} \gpfillpath;
\gpfill{color=gp lt color 1,gp pattern 1,pattern color=.} \gpfillpath;
\draw[gp path] (2.836,1.126)--(2.836,2.203)--(2.967,2.203)--(2.967,1.126)--cycle;
\def\gpfillpath{(3.361,1.126)--(3.493,1.126)--(3.493,2.419)--(3.361,2.419)--cycle}
\gpfill{color=gpbgfillcolor} \gpfillpath;
\gpfill{color=gp lt color 1,gp pattern 1,pattern color=.} \gpfillpath;
\draw[gp path] (3.361,1.126)--(3.361,2.418)--(3.492,2.418)--(3.492,1.126)--cycle;
\def\gpfillpath{(3.886,1.126)--(4.018,1.126)--(4.018,2.376)--(3.886,2.376)--cycle}
\gpfill{color=gpbgfillcolor} \gpfillpath;
\gpfill{color=gp lt color 1,gp pattern 1,pattern color=.} \gpfillpath;
\draw[gp path] (3.886,1.126)--(3.886,2.375)--(4.017,2.375)--(4.017,1.126)--cycle;
\gpcolor{color=gp lt color border}
\gpsetlinetype{gp lt border}
\draw[gp path] (1.196,3.279)--(1.196,1.126)--(4.345,1.126)--(4.345,3.279)--cycle;
%% coordinates of the plot area
\gpdefrectangularnode{gp plot 1}{\pgfpoint{1.196cm}{1.126cm}}{\pgfpoint{4.345cm}{3.279cm}}
\end{tikzpicture}
%% gnuplot variables

	\vspace{-15pt}
	\caption{SpMV}
	\label{fig:spmm1}
\end{subfigure}%
\begin{subfigure}{.5\linewidth}
	\begin{tikzpicture}[gnuplot]
%% generated with GNUPLOT 4.6p4 (Lua 5.1; terminal rev. 99, script rev. 100)
%% Wed 11 Nov 2015 10:11:43 AM EST
\path (0.000,0.000) rectangle (5.588,4.572);
\gpcolor{color=gp lt color border}
\gpsetlinetype{gp lt border}
\gpsetlinewidth{1.00}
\draw[gp path] (1.196,1.126)--(1.376,1.126);
\draw[gp path] (4.345,1.126)--(4.165,1.126);
\node[gp node right] at (1.012,1.126) { 0};
\draw[gp path] (1.196,1.557)--(1.376,1.557);
\draw[gp path] (4.345,1.557)--(4.165,1.557);
\node[gp node right] at (1.012,1.557) { 0.2};
\draw[gp path] (1.196,1.987)--(1.376,1.987);
\draw[gp path] (4.345,1.987)--(4.165,1.987);
\node[gp node right] at (1.012,1.987) { 0.4};
\draw[gp path] (1.196,2.418)--(1.376,2.418);
\draw[gp path] (4.345,2.418)--(4.165,2.418);
\node[gp node right] at (1.012,2.418) { 0.6};
\draw[gp path] (1.196,2.848)--(1.376,2.848);
\draw[gp path] (4.345,2.848)--(4.165,2.848);
\node[gp node right] at (1.012,2.848) { 0.8};
\draw[gp path] (1.196,3.279)--(1.376,3.279);
\draw[gp path] (4.345,3.279)--(4.165,3.279);
\node[gp node right] at (1.012,3.279) { 1};
\draw[gp path] (1.721,1.126)--(1.721,1.306);
\draw[gp path] (1.721,3.279)--(1.721,3.099);
\node[gp node left,rotate=-20] at (1.721,0.942) {rmat-100M-40};
\draw[gp path] (2.246,1.126)--(2.246,1.306);
\draw[gp path] (2.246,3.279)--(2.246,3.099);
\node[gp node left,rotate=-20] at (2.246,0.942) {rmat-100M-160};
\draw[gp path] (2.771,1.126)--(2.771,1.306);
\draw[gp path] (2.771,3.279)--(2.771,3.099);
\node[gp node left,rotate=-20] at (2.771,0.942) {W};
\draw[gp path] (3.295,1.126)--(3.295,1.306);
\draw[gp path] (3.295,3.279)--(3.295,3.099);
\node[gp node left,rotate=-20] at (3.295,0.942) {Friendster};
\draw[gp path] (3.820,1.126)--(3.820,1.306);
\draw[gp path] (3.820,3.279)--(3.820,3.099);
\node[gp node left,rotate=-20] at (3.820,0.942) {Page graph};
\draw[gp path] (1.196,3.279)--(1.196,1.126)--(4.345,1.126)--(4.345,3.279)--cycle;
\node[gp node right] at (2.864,4.238) {Trilinos};
\def\gpfillpath{(3.048,4.161)--(3.964,4.161)--(3.964,4.315)--(3.048,4.315)--cycle}
\gpfill{color=gpbgfillcolor} \gpfillpath;
\gpfill{color=gp lt color 0,gp pattern 0,pattern color=.} \gpfillpath;
\gpcolor{color=gp lt color 0}
\gpsetlinetype{gp lt plot 0}
\draw[gp path] (3.048,4.161)--(3.964,4.161)--(3.964,4.315)--(3.048,4.315)--cycle;
\def\gpfillpath{(1.655,1.126)--(1.787,1.126)--(1.787,2.010)--(1.655,2.010)--cycle}
\gpfill{color=gpbgfillcolor} \gpfillpath;
\gpfill{color=gp lt color 0,gp pattern 0,pattern color=.} \gpfillpath;
\draw[gp path] (1.655,1.126)--(1.655,2.009)--(1.786,2.009)--(1.786,1.126)--cycle;
\def\gpfillpath{(2.180,1.126)--(2.312,1.126)--(2.312,1.881)--(2.180,1.881)--cycle}
\gpfill{color=gpbgfillcolor} \gpfillpath;
\gpfill{color=gp lt color 0,gp pattern 0,pattern color=.} \gpfillpath;
\draw[gp path] (2.180,1.126)--(2.180,1.880)--(2.311,1.880)--(2.311,1.126)--cycle;
\def\gpfillpath{(2.705,1.126)--(2.837,1.126)--(2.837,2.053)--(2.705,2.053)--cycle}
\gpfill{color=gpbgfillcolor} \gpfillpath;
\gpfill{color=gp lt color 0,gp pattern 0,pattern color=.} \gpfillpath;
\draw[gp path] (2.705,1.126)--(2.705,2.052)--(2.836,2.052)--(2.836,1.126)--cycle;
\def\gpfillpath{(3.230,1.126)--(3.362,1.126)--(3.362,2.031)--(3.230,2.031)--cycle}
\gpfill{color=gpbgfillcolor} \gpfillpath;
\gpfill{color=gp lt color 0,gp pattern 0,pattern color=.} \gpfillpath;
\draw[gp path] (3.230,1.126)--(3.230,2.030)--(3.361,2.030)--(3.361,1.126)--cycle;
\gpcolor{color=gp lt color border}
\node[gp node right] at (2.864,3.930) {SEM};
\def\gpfillpath{(3.048,3.853)--(3.964,3.853)--(3.964,4.007)--(3.048,4.007)--cycle}
\gpfill{color=gpbgfillcolor} \gpfillpath;
\gpfill{color=gp lt color 1,gp pattern 1,pattern color=.} \gpfillpath;
\gpcolor{color=gp lt color 1}
\gpsetlinetype{gp lt plot 1}
\draw[gp path] (3.048,3.853)--(3.964,3.853)--(3.964,4.007)--(3.048,4.007)--cycle;
\def\gpfillpath{(1.786,1.126)--(1.919,1.126)--(1.919,2.979)--(1.786,2.979)--cycle}
\gpfill{color=gpbgfillcolor} \gpfillpath;
\gpfill{color=gp lt color 1,gp pattern 1,pattern color=.} \gpfillpath;
\draw[gp path] (1.786,1.126)--(1.786,2.978)--(1.918,2.978)--(1.918,1.126)--cycle;
\def\gpfillpath{(2.311,1.126)--(2.443,1.126)--(2.443,2.828)--(2.311,2.828)--cycle}
\gpfill{color=gpbgfillcolor} \gpfillpath;
\gpfill{color=gp lt color 1,gp pattern 1,pattern color=.} \gpfillpath;
\draw[gp path] (2.311,1.126)--(2.311,2.827)--(2.442,2.827)--(2.442,1.126)--cycle;
\def\gpfillpath{(2.836,1.126)--(2.968,1.126)--(2.968,2.785)--(2.836,2.785)--cycle}
\gpfill{color=gpbgfillcolor} \gpfillpath;
\gpfill{color=gp lt color 1,gp pattern 1,pattern color=.} \gpfillpath;
\draw[gp path] (2.836,1.126)--(2.836,2.784)--(2.967,2.784)--(2.967,1.126)--cycle;
\def\gpfillpath{(3.361,1.126)--(3.493,1.126)--(3.493,3.043)--(3.361,3.043)--cycle}
\gpfill{color=gpbgfillcolor} \gpfillpath;
\gpfill{color=gp lt color 1,gp pattern 1,pattern color=.} \gpfillpath;
\draw[gp path] (3.361,1.126)--(3.361,3.042)--(3.492,3.042)--(3.492,1.126)--cycle;
\def\gpfillpath{(3.886,1.126)--(4.018,1.126)--(4.018,2.806)--(3.886,2.806)--cycle}
\gpfill{color=gpbgfillcolor} \gpfillpath;
\gpfill{color=gp lt color 1,gp pattern 1,pattern color=.} \gpfillpath;
\draw[gp path] (3.886,1.126)--(3.886,2.805)--(4.017,2.805)--(4.017,1.126)--cycle;
\gpcolor{color=gp lt color border}
\gpsetlinetype{gp lt border}
\draw[gp path] (1.196,3.279)--(1.196,1.126)--(4.345,1.126)--(4.345,3.279)--cycle;
%% coordinates of the plot area
\gpdefrectangularnode{gp plot 1}{\pgfpoint{1.196cm}{1.126cm}}{\pgfpoint{4.345cm}{3.279cm}}
\end{tikzpicture}
%% gnuplot variables

	\vspace{-15pt}
	\caption{SpMM}
	\label{fig:spmm4}
\end{subfigure}
\caption{The performance of Trilinos and FlashEigen-SEM sparse matrix
multiplication relative to FlashEigen-IM sparse matrix multiplication.
In sparse matrix dense multiplication (SpMM), there are four columns
in the dense matrix.}
\vspace{-15pt}
\label{fig:spmm}
\end{figure}

%\begin{figure}
%	\footnotesize
%	\begin{subfigure}{.2\textwidth}
%		\begin{tikzpicture}[gnuplot]
%% generated with GNUPLOT 4.6p4 (Lua 5.1; terminal rev. 99, script rev. 100)
%% Tue 06 Oct 2015 10:20:02 PM EDT
\path (0.000,0.000) rectangle (8.382,4.572);
\gpcolor{color=gp lt color border}
\gpsetlinetype{gp lt border}
\gpsetlinewidth{1.00}
\draw[gp path] (1.320,0.985)--(1.500,0.985);
\draw[gp path] (7.829,0.985)--(7.649,0.985);
\node[gp node right] at (1.136,0.985) { 0};
\draw[gp path] (1.320,1.313)--(1.500,1.313);
\draw[gp path] (7.829,1.313)--(7.649,1.313);
\node[gp node right] at (1.136,1.313) { 5};
\draw[gp path] (1.320,1.640)--(1.500,1.640);
\draw[gp path] (7.829,1.640)--(7.649,1.640);
\node[gp node right] at (1.136,1.640) { 10};
\draw[gp path] (1.320,1.968)--(1.500,1.968);
\draw[gp path] (7.829,1.968)--(7.649,1.968);
\node[gp node right] at (1.136,1.968) { 15};
\draw[gp path] (1.320,2.296)--(1.500,2.296);
\draw[gp path] (7.829,2.296)--(7.649,2.296);
\node[gp node right] at (1.136,2.296) { 20};
\draw[gp path] (1.320,2.624)--(1.500,2.624);
\draw[gp path] (7.829,2.624)--(7.649,2.624);
\node[gp node right] at (1.136,2.624) { 25};
\draw[gp path] (1.320,2.951)--(1.500,2.951);
\draw[gp path] (7.829,2.951)--(7.649,2.951);
\node[gp node right] at (1.136,2.951) { 30};
\draw[gp path] (1.320,3.279)--(1.500,3.279);
\draw[gp path] (7.829,3.279)--(7.649,3.279);
\node[gp node right] at (1.136,3.279) { 35};
\draw[gp path] (1.320,0.985)--(1.320,1.165);
\draw[gp path] (1.320,3.279)--(1.320,3.099);
\node[gp node center] at (1.320,0.677) {1};
\draw[gp path] (2.947,0.985)--(2.947,1.165);
\draw[gp path] (2.947,3.279)--(2.947,3.099);
\node[gp node center] at (2.947,0.677) {2};
\draw[gp path] (4.575,0.985)--(4.575,1.165);
\draw[gp path] (4.575,3.279)--(4.575,3.099);
\node[gp node center] at (4.575,0.677) {4};
\draw[gp path] (6.202,0.985)--(6.202,1.165);
\draw[gp path] (6.202,3.279)--(6.202,3.099);
\node[gp node center] at (6.202,0.677) {8};
\draw[gp path] (7.829,0.985)--(7.829,1.165);
\draw[gp path] (7.829,3.279)--(7.829,3.099);
\node[gp node center] at (7.829,0.677) {16};
\draw[gp path] (1.320,3.279)--(1.320,0.985)--(7.829,0.985)--(7.829,3.279)--cycle;
\node[gp node center,rotate=-270] at (0.246,2.132) {seconds};
\node[gp node center] at (4.574,0.215) {The number of columns in the dense matrix};
\node[gp node right] at (3.290,4.238) {FE-IM};
\gpcolor{color=gp lt color 0}
\gpsetlinetype{gp lt plot 0}
\draw[gp path] (3.474,4.238)--(4.390,4.238);
\draw[gp path] (1.320,1.094)--(2.947,1.127)--(4.575,1.199)--(6.202,1.311)--(7.829,1.573);
\gpsetpointsize{4.00}
\gppoint{gp mark 1}{(1.320,1.094)}
\gppoint{gp mark 1}{(2.947,1.127)}
\gppoint{gp mark 1}{(4.575,1.199)}
\gppoint{gp mark 1}{(6.202,1.311)}
\gppoint{gp mark 1}{(7.829,1.573)}
\gppoint{gp mark 1}{(3.932,4.238)}
\gpcolor{color=gp lt color border}
\node[gp node right] at (3.290,3.930) {FE-SEM};
\gpcolor{color=gp lt color 1}
\gpsetlinetype{gp lt plot 1}
\draw[gp path] (3.474,3.930)--(4.390,3.930);
\draw[gp path] (1.320,1.150)--(2.947,1.158)--(4.575,1.242)--(6.202,1.323)--(7.829,1.628);
\gppoint{gp mark 2}{(1.320,1.150)}
\gppoint{gp mark 2}{(2.947,1.158)}
\gppoint{gp mark 2}{(4.575,1.242)}
\gppoint{gp mark 2}{(6.202,1.323)}
\gppoint{gp mark 2}{(7.829,1.628)}
\gppoint{gp mark 2}{(3.932,3.930)}
\gpcolor{color=gp lt color border}
\node[gp node right] at (6.046,4.238) {MKL};
\gpcolor{color=gp lt color 2}
\gpsetlinetype{gp lt plot 2}
\draw[gp path] (6.230,4.238)--(7.146,4.238);
\draw[gp path] (1.320,1.384)--(2.947,1.389)--(4.575,1.570)--(6.202,1.689)--(7.829,1.973);
\gppoint{gp mark 3}{(1.320,1.384)}
\gppoint{gp mark 3}{(2.947,1.389)}
\gppoint{gp mark 3}{(4.575,1.570)}
\gppoint{gp mark 3}{(6.202,1.689)}
\gppoint{gp mark 3}{(7.829,1.973)}
\gppoint{gp mark 3}{(6.688,4.238)}
\gpcolor{color=gp lt color border}
\node[gp node right] at (6.046,3.930) {Trilinos};
\gpcolor{color=gp lt color 3}
\gpsetlinetype{gp lt plot 3}
\draw[gp path] (6.230,3.930)--(7.146,3.930);
\draw[gp path] (1.320,1.155)--(2.947,1.281)--(4.575,1.530)--(6.202,2.036)--(7.829,3.126);
\gppoint{gp mark 4}{(1.320,1.155)}
\gppoint{gp mark 4}{(2.947,1.281)}
\gppoint{gp mark 4}{(4.575,1.530)}
\gppoint{gp mark 4}{(6.202,2.036)}
\gppoint{gp mark 4}{(7.829,3.126)}
\gppoint{gp mark 4}{(6.688,3.930)}
\gpcolor{color=gp lt color border}
\gpsetlinetype{gp lt border}
\draw[gp path] (1.320,3.279)--(1.320,0.985)--(7.829,0.985)--(7.829,3.279)--cycle;
%% coordinates of the plot area
\gpdefrectangularnode{gp plot 1}{\pgfpoint{1.320cm}{0.985cm}}{\pgfpoint{7.829cm}{3.279cm}}
\end{tikzpicture}
%% gnuplot variables

%		\caption{A subfigure}
%		\label{fig:sub1}
%	\end{subfigure}%
%	\begin{subfigure}{.2\textwidth}
%		\begin{tikzpicture}[gnuplot]
%% generated with GNUPLOT 4.6p4 (Lua 5.1; terminal rev. 99, script rev. 100)
%% Fri 02 Oct 2015 11:19:03 PM EDT
\path (0.000,0.000) rectangle (8.382,4.572);
\gpcolor{color=gp lt color border}
\gpsetlinetype{gp lt border}
\gpsetlinewidth{1.00}
\draw[gp path] (1.320,0.985)--(1.500,0.985);
\draw[gp path] (7.829,0.985)--(7.649,0.985);
\node[gp node right] at (1.136,0.985) { 0};
\draw[gp path] (1.320,1.240)--(1.500,1.240);
\draw[gp path] (7.829,1.240)--(7.649,1.240);
\node[gp node right] at (1.136,1.240) { 5};
\draw[gp path] (1.320,1.495)--(1.500,1.495);
\draw[gp path] (7.829,1.495)--(7.649,1.495);
\node[gp node right] at (1.136,1.495) { 10};
\draw[gp path] (1.320,1.750)--(1.500,1.750);
\draw[gp path] (7.829,1.750)--(7.649,1.750);
\node[gp node right] at (1.136,1.750) { 15};
\draw[gp path] (1.320,2.005)--(1.500,2.005);
\draw[gp path] (7.829,2.005)--(7.649,2.005);
\node[gp node right] at (1.136,2.005) { 20};
\draw[gp path] (1.320,2.259)--(1.500,2.259);
\draw[gp path] (7.829,2.259)--(7.649,2.259);
\node[gp node right] at (1.136,2.259) { 25};
\draw[gp path] (1.320,2.514)--(1.500,2.514);
\draw[gp path] (7.829,2.514)--(7.649,2.514);
\node[gp node right] at (1.136,2.514) { 30};
\draw[gp path] (1.320,2.769)--(1.500,2.769);
\draw[gp path] (7.829,2.769)--(7.649,2.769);
\node[gp node right] at (1.136,2.769) { 35};
\draw[gp path] (1.320,3.024)--(1.500,3.024);
\draw[gp path] (7.829,3.024)--(7.649,3.024);
\node[gp node right] at (1.136,3.024) { 40};
\draw[gp path] (1.320,3.279)--(1.500,3.279);
\draw[gp path] (7.829,3.279)--(7.649,3.279);
\node[gp node right] at (1.136,3.279) { 45};
\draw[gp path] (1.320,0.985)--(1.320,1.165);
\draw[gp path] (1.320,3.279)--(1.320,3.099);
\node[gp node center] at (1.320,0.677) {1};
\draw[gp path] (2.947,0.985)--(2.947,1.165);
\draw[gp path] (2.947,3.279)--(2.947,3.099);
\node[gp node center] at (2.947,0.677) {2};
\draw[gp path] (4.575,0.985)--(4.575,1.165);
\draw[gp path] (4.575,3.279)--(4.575,3.099);
\node[gp node center] at (4.575,0.677) {4};
\draw[gp path] (6.202,0.985)--(6.202,1.165);
\draw[gp path] (6.202,3.279)--(6.202,3.099);
\node[gp node center] at (6.202,0.677) {8};
\draw[gp path] (7.829,0.985)--(7.829,1.165);
\draw[gp path] (7.829,3.279)--(7.829,3.099);
\node[gp node center] at (7.829,0.677) {16};
\draw[gp path] (1.320,3.279)--(1.320,0.985)--(7.829,0.985)--(7.829,3.279)--cycle;
\node[gp node center,rotate=-270] at (0.246,2.132) {seconds};
\node[gp node center] at (4.574,0.215) {The number of columns};
\node[gp node right] at (3.290,4.238) {FE-IM};
\gpcolor{color=gp lt color 0}
\gpsetlinetype{gp lt plot 0}
\draw[gp path] (3.474,4.238)--(4.390,4.238);
\draw[gp path] (1.320,1.140)--(2.947,1.212)--(4.575,1.325)--(6.202,1.596)--(7.829,2.108);
\gpsetpointsize{4.00}
\gppoint{gp mark 1}{(1.320,1.140)}
\gppoint{gp mark 1}{(2.947,1.212)}
\gppoint{gp mark 1}{(4.575,1.325)}
\gppoint{gp mark 1}{(6.202,1.596)}
\gppoint{gp mark 1}{(7.829,2.108)}
\gppoint{gp mark 1}{(3.932,4.238)}
\gpcolor{color=gp lt color border}
\node[gp node right] at (3.290,3.930) {FE-SEM};
\gpcolor{color=gp lt color 1}
\gpsetlinetype{gp lt plot 1}
\draw[gp path] (3.474,3.930)--(4.390,3.930);
\draw[gp path] (1.320,1.205)--(2.947,1.295)--(4.575,1.383)--(6.202,1.652)--(7.829,2.164);
\gppoint{gp mark 2}{(1.320,1.205)}
\gppoint{gp mark 2}{(2.947,1.295)}
\gppoint{gp mark 2}{(4.575,1.383)}
\gppoint{gp mark 2}{(6.202,1.652)}
\gppoint{gp mark 2}{(7.829,2.164)}
\gppoint{gp mark 2}{(3.932,3.930)}
\gpcolor{color=gp lt color border}
\node[gp node right] at (6.046,4.238) {MKL};
\gpcolor{color=gp lt color 2}
\gpsetlinetype{gp lt plot 2}
\draw[gp path] (6.230,4.238)--(7.146,4.238);
\draw[gp path] (1.320,1.653)--(2.947,1.748)--(4.575,2.051)--(6.202,2.275)--(7.829,3.166);
\gppoint{gp mark 3}{(1.320,1.653)}
\gppoint{gp mark 3}{(2.947,1.748)}
\gppoint{gp mark 3}{(4.575,2.051)}
\gppoint{gp mark 3}{(6.202,2.275)}
\gppoint{gp mark 3}{(7.829,3.166)}
\gppoint{gp mark 3}{(6.688,4.238)}
\gpcolor{color=gp lt color border}
\node[gp node right] at (6.046,3.930) {Trilinos};
\gpcolor{color=gp lt color 3}
\gpsetlinetype{gp lt plot 3}
\draw[gp path] (6.230,3.930)--(7.146,3.930);
\draw[gp path] (1.320,1.371)--(2.947,1.532)--(4.575,1.999)--(6.202,2.952);
\gppoint{gp mark 4}{(1.320,1.371)}
\gppoint{gp mark 4}{(2.947,1.532)}
\gppoint{gp mark 4}{(4.575,1.999)}
\gppoint{gp mark 4}{(6.202,2.952)}
\gppoint{gp mark 4}{(6.688,3.930)}
\gpcolor{color=gp lt color border}
\gpsetlinetype{gp lt border}
\draw[gp path] (1.320,3.279)--(1.320,0.985)--(7.829,0.985)--(7.829,3.279)--cycle;
%% coordinates of the plot area
\gpdefrectangularnode{gp plot 1}{\pgfpoint{1.320cm}{0.985cm}}{\pgfpoint{7.829cm}{3.279cm}}
\end{tikzpicture}
%% gnuplot variables

%		\caption{A subfigure}
%		\label{fig:sub2}
%	\end{subfigure}
%	\caption{A figure with two subfigures}
%	\label{fig:test}
%\end{figure}

\subsection{Dense matrix multiplication}

This section evaluates the performance of dense matrix multiplication,
the other computationally intensive operation in an eigensolver.
For dense matrix multiplication, we focus on I/O optimizations, so we evaluate
the effectiveness of the I/O optimizations on this operation. Then we
compare the performance of our in-memory and external-memory
implementations with the ones in Intel MKL and Trilinos.

In eigendecomposition, there are two forms of dense matrix multiplication.
The first form, shown by $op1$ in Table \ref{anasazi_ops}, multiplies
a tall-and-skinny dense matrix of $n \times m$
with a small dense matrix of $m \times b$ and outputs a tall-and-skinny dense
matrix of $n \times b$, where $n$ is the size of the eigenvalue problem,
$m$ is the number of existing vectors in the subspace and $b$ is the block
size of the eigensolver. The second form, shown by $op3$, multiplies
a transpose of a tall-and-skinny dense matrix of $n \times m$ with another
tall-and-skinny dense matrix of $n \times b$ and outputs a small matrix.
In both forms, $m$ varies from one block size to the maximal subspace size
specified by a user and is increased by one block size in each
iteration. In the experiments, we set $n$ to 60M, roughly the size of
eigenvalue problems in Table \ref{graphs}, $b$ to 4 and vary $m$
from $4$ to $512$. This setting of $b$ and $m$ is used in the experiments
of our external-memory KrylovSchur eigensolver in the next section.

\subsubsection{Optimizations on dense matrix multiplication}

We illustrate the most effective I/O optimizations on dense matrix
multiplication. We start with a basic implementation of matrix multiplication
in SAFS and deploy the optimizations on SAFS and matrix multiplication
incrementally in the following order:
\begin{itemize} \itemsep1pt \parskip0pt \parsep0pt
	\item use different random striping orders for each file (\textit{diff strip}),
	\item use a per-thread buffer pool to allocate memory for I/O (\textit{buf pool}),
	\item use one I/O thread per NUMA node (\textit{1IOT}),
	\item use I/O polling in worker threads (\textit{polling}),
	\item use 8MB for the maximal block size in the kernel (\textit{max block}),
\end{itemize}

The effectiveness of the optimizations on external memory dense matrix
multiplication is shown in Figure \ref{perf:dmm_opts}. We only demonstrate
their effectiveness on the second form of dense matrix multiplication and
their effectiveness in the first form is very similar. As shown in Figure
\ref{perf:dmm_opts}, using a per-thread buffer pool and a smaller number
of I/O threads has the most significant performance improvement. By combining
all of the optimizations together, we increase the performance by a factor
of up to four. This also indicates the importance of reducing the overhead
of thread context switches in high-throughput I/O access.

\begin{figure}
	\begin{center}
		\footnotesize
		\vspace{-15pt}
		\begin{tikzpicture}[gnuplot]
%% generated with GNUPLOT 4.6p4 (Lua 5.1; terminal rev. 99, script rev. 100)
%% Sat 03 Oct 2015 04:23:12 PM EDT
\path (0.000,0.000) rectangle (8.382,4.572);
\gpcolor{color=gp lt color border}
\gpsetlinetype{gp lt border}
\gpsetlinewidth{1.00}
\draw[gp path] (1.320,0.616)--(1.500,0.616);
\draw[gp path] (7.829,0.616)--(7.649,0.616);
\node[gp node right] at (1.136,0.616) { 0};
\draw[gp path] (1.320,0.878)--(1.500,0.878);
\draw[gp path] (7.829,0.878)--(7.649,0.878);
\node[gp node right] at (1.136,0.878) { 5};
\draw[gp path] (1.320,1.139)--(1.500,1.139);
\draw[gp path] (7.829,1.139)--(7.649,1.139);
\node[gp node right] at (1.136,1.139) { 10};
\draw[gp path] (1.320,1.401)--(1.500,1.401);
\draw[gp path] (7.829,1.401)--(7.649,1.401);
\node[gp node right] at (1.136,1.401) { 15};
\draw[gp path] (1.320,1.663)--(1.500,1.663);
\draw[gp path] (7.829,1.663)--(7.649,1.663);
\node[gp node right] at (1.136,1.663) { 20};
\draw[gp path] (1.320,1.924)--(1.500,1.924);
\draw[gp path] (7.829,1.924)--(7.649,1.924);
\node[gp node right] at (1.136,1.924) { 25};
\draw[gp path] (1.320,2.186)--(1.500,2.186);
\draw[gp path] (7.829,2.186)--(7.649,2.186);
\node[gp node right] at (1.136,2.186) { 30};
\draw[gp path] (1.320,2.448)--(1.500,2.448);
\draw[gp path] (7.829,2.448)--(7.649,2.448);
\node[gp node right] at (1.136,2.448) { 35};
\draw[gp path] (1.320,2.709)--(1.500,2.709);
\draw[gp path] (7.829,2.709)--(7.649,2.709);
\node[gp node right] at (1.136,2.709) { 40};
\draw[gp path] (1.320,2.971)--(1.500,2.971);
\draw[gp path] (7.829,2.971)--(7.649,2.971);
\node[gp node right] at (1.136,2.971) { 45};
\draw[gp path] (3.490,0.616)--(3.490,0.796);
\draw[gp path] (3.490,2.971)--(3.490,2.791);
\node[gp node center] at (3.490,0.308) {MvTransMv-16};
\draw[gp path] (5.659,0.616)--(5.659,0.796);
\draw[gp path] (5.659,2.971)--(5.659,2.791);
\node[gp node center] at (5.659,0.308) {MvTransMv-64};
\draw[gp path] (1.320,2.971)--(1.320,0.616)--(7.829,0.616)--(7.829,2.971)--cycle;
\node[gp node center,rotate=-270] at (0.246,1.793) {seconds};
\node[gp node right] at (3.290,4.238) {Base};
\def\gpfillpath{(3.474,4.161)--(4.390,4.161)--(4.390,4.315)--(3.474,4.315)--cycle}
\gpfill{color=gpbgfillcolor} \gpfillpath;
\gpfill{color=gp lt color 0,gp pattern 0,pattern color=.} \gpfillpath;
\gpcolor{color=gp lt color 0}
\gpsetlinetype{gp lt plot 0}
\draw[gp path] (3.474,4.161)--(4.390,4.161)--(4.390,4.315)--(3.474,4.315)--cycle;
\def\gpfillpath{(2.812,0.616)--(3.084,0.616)--(3.084,1.207)--(2.812,1.207)--cycle}
\gpfill{color=gpbgfillcolor} \gpfillpath;
\gpfill{color=gp lt color 0,gp pattern 0,pattern color=.} \gpfillpath;
\draw[gp path] (2.812,0.616)--(2.812,1.206)--(3.083,1.206)--(3.083,0.616)--cycle;
\def\gpfillpath{(4.981,0.616)--(5.254,0.616)--(5.254,2.728)--(4.981,2.728)--cycle}
\gpfill{color=gpbgfillcolor} \gpfillpath;
\gpfill{color=gp lt color 0,gp pattern 0,pattern color=.} \gpfillpath;
\draw[gp path] (4.981,0.616)--(4.981,2.727)--(5.253,2.727)--(5.253,0.616)--cycle;
\gpcolor{color=gp lt color border}
\node[gp node right] at (3.290,3.930) {diff stripe};
\def\gpfillpath{(3.474,3.853)--(4.390,3.853)--(4.390,4.007)--(3.474,4.007)--cycle}
\gpfill{color=gpbgfillcolor} \gpfillpath;
\gpfill{color=gp lt color 1,gp pattern 1,pattern color=.} \gpfillpath;
\gpcolor{color=gp lt color 1}
\gpsetlinetype{gp lt plot 1}
\draw[gp path] (3.474,3.853)--(4.390,3.853)--(4.390,4.007)--(3.474,4.007)--cycle;
\def\gpfillpath{(3.083,0.616)--(3.355,0.616)--(3.355,1.162)--(3.083,1.162)--cycle}
\gpfill{color=gpbgfillcolor} \gpfillpath;
\gpfill{color=gp lt color 1,gp pattern 1,pattern color=.} \gpfillpath;
\draw[gp path] (3.083,0.616)--(3.083,1.161)--(3.354,1.161)--(3.354,0.616)--cycle;
\def\gpfillpath{(5.253,0.616)--(5.525,0.616)--(5.525,2.603)--(5.253,2.603)--cycle}
\gpfill{color=gpbgfillcolor} \gpfillpath;
\gpfill{color=gp lt color 1,gp pattern 1,pattern color=.} \gpfillpath;
\draw[gp path] (5.253,0.616)--(5.253,2.602)--(5.524,2.602)--(5.524,0.616)--cycle;
\gpcolor{color=gp lt color border}
\node[gp node right] at (3.290,3.622) {buf pool};
\def\gpfillpath{(3.474,3.545)--(4.390,3.545)--(4.390,3.699)--(3.474,3.699)--cycle}
\gpfill{color=gpbgfillcolor} \gpfillpath;
\gpfill{color=gp lt color 2,gp pattern 2,pattern color=.} \gpfillpath;
\gpcolor{color=gp lt color 2}
\gpsetlinetype{gp lt plot 2}
\draw[gp path] (3.474,3.545)--(4.390,3.545)--(4.390,3.699)--(3.474,3.699)--cycle;
\def\gpfillpath{(3.354,0.616)--(3.626,0.616)--(3.626,0.958)--(3.354,0.958)--cycle}
\gpfill{color=gpbgfillcolor} \gpfillpath;
\gpfill{color=gp lt color 2,gp pattern 2,pattern color=.} \gpfillpath;
\draw[gp path] (3.354,0.616)--(3.354,0.957)--(3.625,0.957)--(3.625,0.616)--cycle;
\def\gpfillpath{(5.524,0.616)--(5.796,0.616)--(5.796,2.119)--(5.524,2.119)--cycle}
\gpfill{color=gpbgfillcolor} \gpfillpath;
\gpfill{color=gp lt color 2,gp pattern 2,pattern color=.} \gpfillpath;
\draw[gp path] (5.524,0.616)--(5.524,2.118)--(5.795,2.118)--(5.795,0.616)--cycle;
\gpcolor{color=gp lt color border}
\node[gp node right] at (6.598,4.238) {1IOT};
\def\gpfillpath{(6.782,4.161)--(7.698,4.161)--(7.698,4.315)--(6.782,4.315)--cycle}
\gpfill{color=gpbgfillcolor} \gpfillpath;
\gpfill{color=gp lt color 3,gp pattern 3,pattern color=.} \gpfillpath;
\gpcolor{color=gp lt color 3}
\gpsetlinetype{gp lt plot 3}
\draw[gp path] (6.782,4.161)--(7.698,4.161)--(7.698,4.315)--(6.782,4.315)--cycle;
\def\gpfillpath{(3.625,0.616)--(3.897,0.616)--(3.897,0.849)--(3.625,0.849)--cycle}
\gpfill{color=gpbgfillcolor} \gpfillpath;
\gpfill{color=gp lt color 3,gp pattern 3,pattern color=.} \gpfillpath;
\draw[gp path] (3.625,0.616)--(3.625,0.848)--(3.896,0.848)--(3.896,0.616)--cycle;
\def\gpfillpath{(5.795,0.616)--(6.067,0.616)--(6.067,1.545)--(5.795,1.545)--cycle}
\gpfill{color=gpbgfillcolor} \gpfillpath;
\gpfill{color=gp lt color 3,gp pattern 3,pattern color=.} \gpfillpath;
\draw[gp path] (5.795,0.616)--(5.795,1.544)--(6.066,1.544)--(6.066,0.616)--cycle;
\gpcolor{color=gp lt color border}
\node[gp node right] at (6.598,3.930) {polling};
\def\gpfillpath{(6.782,3.853)--(7.698,3.853)--(7.698,4.007)--(6.782,4.007)--cycle}
\gpfill{color=gpbgfillcolor} \gpfillpath;
\gpfill{color=gp lt color 4,gp pattern 4,pattern color=.} \gpfillpath;
\gpcolor{color=gp lt color 4}
\gpsetlinetype{gp lt plot 4}
\draw[gp path] (6.782,3.853)--(7.698,3.853)--(7.698,4.007)--(6.782,4.007)--cycle;
\def\gpfillpath{(3.896,0.616)--(4.169,0.616)--(4.169,0.806)--(3.896,0.806)--cycle}
\gpfill{color=gpbgfillcolor} \gpfillpath;
\gpfill{color=gp lt color 4,gp pattern 4,pattern color=.} \gpfillpath;
\draw[gp path] (3.896,0.616)--(3.896,0.805)--(4.168,0.805)--(4.168,0.616)--cycle;
\def\gpfillpath{(6.066,0.616)--(6.338,0.616)--(6.338,1.308)--(6.066,1.308)--cycle}
\gpfill{color=gpbgfillcolor} \gpfillpath;
\gpfill{color=gp lt color 4,gp pattern 4,pattern color=.} \gpfillpath;
\draw[gp path] (6.066,0.616)--(6.066,1.307)--(6.337,1.307)--(6.337,0.616)--cycle;
\gpcolor{color=gp lt color border}
\node[gp node right] at (6.598,3.622) {max block};
\def\gpfillpath{(6.782,3.545)--(7.698,3.545)--(7.698,3.699)--(6.782,3.699)--cycle}
\gpfill{color=gpbgfillcolor} \gpfillpath;
\gpfill{color=gp lt color 5,gp pattern 5,pattern color=.} \gpfillpath;
\gpcolor{color=gp lt color 5}
\gpsetlinetype{gp lt plot 5}
\draw[gp path] (6.782,3.545)--(7.698,3.545)--(7.698,3.699)--(6.782,3.699)--cycle;
\def\gpfillpath{(4.168,0.616)--(4.440,0.616)--(4.440,0.811)--(4.168,0.811)--cycle}
\gpfill{color=gpbgfillcolor} \gpfillpath;
\gpfill{color=gp lt color 5,gp pattern 5,pattern color=.} \gpfillpath;
\draw[gp path] (4.168,0.616)--(4.168,0.810)--(4.439,0.810)--(4.439,0.616)--cycle;
\def\gpfillpath{(6.337,0.616)--(6.610,0.616)--(6.610,1.239)--(6.337,1.239)--cycle}
\gpfill{color=gpbgfillcolor} \gpfillpath;
\gpfill{color=gp lt color 5,gp pattern 5,pattern color=.} \gpfillpath;
\draw[gp path] (6.337,0.616)--(6.337,1.238)--(6.609,1.238)--(6.609,0.616)--cycle;
\gpcolor{color=gp lt color border}
\gpsetlinetype{gp lt border}
\draw[gp path] (1.320,2.971)--(1.320,0.616)--(7.829,0.616)--(7.829,2.971)--cycle;
%% coordinates of the plot area
\gpdefrectangularnode{gp plot 1}{\pgfpoint{1.320cm}{0.616cm}}{\pgfpoint{7.829cm}{2.971cm}}
\end{tikzpicture}
%% gnuplot variables

		\vspace{-15pt}
		\caption{The effectiveness of I/O optimizations on dense matrix
		multiplication.}
		\label{perf:dmm_opts}
	\end{center}
\end{figure}

\subsubsection{Dense matrix multiplication performance}

\begin{figure}
	\begin{center}
		\footnotesize
		\vspace{-15pt}
		\begin{tikzpicture}[gnuplot]
%% generated with GNUPLOT 4.6p4 (Lua 5.1; terminal rev. 99, script rev. 100)
%% Sat 03 Oct 2015 04:01:21 PM EDT
\path (0.000,0.000) rectangle (8.382,4.572);
\gpcolor{color=gp lt color border}
\gpsetlinetype{gp lt border}
\gpsetlinewidth{1.00}
\draw[gp path] (2.424,0.985)--(2.604,0.985);
\draw[gp path] (7.829,0.985)--(7.649,0.985);
\node[gp node right] at (2.240,0.985) { 0.015625};
\draw[gp path] (2.424,1.176)--(2.514,1.176);
\draw[gp path] (7.829,1.176)--(7.739,1.176);
\draw[gp path] (2.424,1.367)--(2.604,1.367);
\draw[gp path] (7.829,1.367)--(7.649,1.367);
\node[gp node right] at (2.240,1.367) { 0.0625};
\draw[gp path] (2.424,1.559)--(2.514,1.559);
\draw[gp path] (7.829,1.559)--(7.739,1.559);
\draw[gp path] (2.424,1.750)--(2.604,1.750);
\draw[gp path] (7.829,1.750)--(7.649,1.750);
\node[gp node right] at (2.240,1.750) { 0.25};
\draw[gp path] (2.424,1.941)--(2.514,1.941);
\draw[gp path] (7.829,1.941)--(7.739,1.941);
\draw[gp path] (2.424,2.132)--(2.604,2.132);
\draw[gp path] (7.829,2.132)--(7.649,2.132);
\node[gp node right] at (2.240,2.132) { 1};
\draw[gp path] (2.424,2.323)--(2.514,2.323);
\draw[gp path] (7.829,2.323)--(7.739,2.323);
\draw[gp path] (2.424,2.514)--(2.604,2.514);
\draw[gp path] (7.829,2.514)--(7.649,2.514);
\node[gp node right] at (2.240,2.514) { 4};
\draw[gp path] (2.424,2.706)--(2.514,2.706);
\draw[gp path] (7.829,2.706)--(7.739,2.706);
\draw[gp path] (2.424,2.897)--(2.604,2.897);
\draw[gp path] (7.829,2.897)--(7.649,2.897);
\node[gp node right] at (2.240,2.897) { 16};
\draw[gp path] (2.424,3.088)--(2.514,3.088);
\draw[gp path] (7.829,3.088)--(7.739,3.088);
\draw[gp path] (2.424,3.279)--(2.604,3.279);
\draw[gp path] (7.829,3.279)--(7.649,3.279);
\node[gp node right] at (2.240,3.279) { 64};
\draw[gp path] (2.424,0.985)--(2.424,1.165);
\draw[gp path] (2.424,3.279)--(2.424,3.099);
\node[gp node center] at (2.424,0.677) {4};
\draw[gp path] (3.196,0.985)--(3.196,1.165);
\draw[gp path] (3.196,3.279)--(3.196,3.099);
\node[gp node center] at (3.196,0.677) {8};
\draw[gp path] (3.968,0.985)--(3.968,1.165);
\draw[gp path] (3.968,3.279)--(3.968,3.099);
\node[gp node center] at (3.968,0.677) {16};
\draw[gp path] (4.740,0.985)--(4.740,1.165);
\draw[gp path] (4.740,3.279)--(4.740,3.099);
\node[gp node center] at (4.740,0.677) {32};
\draw[gp path] (5.513,0.985)--(5.513,1.165);
\draw[gp path] (5.513,3.279)--(5.513,3.099);
\node[gp node center] at (5.513,0.677) {64};
\draw[gp path] (6.285,0.985)--(6.285,1.165);
\draw[gp path] (6.285,3.279)--(6.285,3.099);
\node[gp node center] at (6.285,0.677) {128};
\draw[gp path] (7.057,0.985)--(7.057,1.165);
\draw[gp path] (7.057,3.279)--(7.057,3.099);
\node[gp node center] at (7.057,0.677) {256};
\draw[gp path] (7.829,0.985)--(7.829,1.165);
\draw[gp path] (7.829,3.279)--(7.829,3.099);
\node[gp node center] at (7.829,0.677) {512};
\draw[gp path] (2.424,3.279)--(2.424,0.985)--(7.829,0.985)--(7.829,3.279)--cycle;
\node[gp node center,rotate=-270] at (0.246,2.132) {seconds};
\node[gp node center] at (5.126,0.215) {The number of columns};
\node[gp node right] at (3.842,4.238) {FE-IM};
\gpcolor{color=gp lt color 0}
\gpsetlinetype{gp lt plot 0}
\draw[gp path] (4.026,4.238)--(4.942,4.238);
\draw[gp path] (2.424,1.569)--(3.196,1.609)--(3.968,1.738)--(4.740,1.941)--(5.513,1.957)%
  --(6.285,2.209)--(7.057,2.398)--(7.829,2.590);
\gpsetpointsize{4.00}
\gppoint{gp mark 1}{(2.424,1.569)}
\gppoint{gp mark 1}{(3.196,1.609)}
\gppoint{gp mark 1}{(3.968,1.738)}
\gppoint{gp mark 1}{(4.740,1.941)}
\gppoint{gp mark 1}{(5.513,1.957)}
\gppoint{gp mark 1}{(6.285,2.209)}
\gppoint{gp mark 1}{(7.057,2.398)}
\gppoint{gp mark 1}{(7.829,2.590)}
\gppoint{gp mark 1}{(4.484,4.238)}
\gpcolor{color=gp lt color border}
\node[gp node right] at (3.842,3.930) {FE-SEM};
\gpcolor{color=gp lt color 1}
\gpsetlinetype{gp lt plot 1}
\draw[gp path] (4.026,3.930)--(4.942,3.930);
\draw[gp path] (2.424,2.026)--(3.196,2.118)--(3.968,2.161)--(4.740,2.282)--(5.513,2.460)%
  --(6.285,2.629)--(7.057,2.813)--(7.829,2.999);
\gppoint{gp mark 2}{(2.424,2.026)}
\gppoint{gp mark 2}{(3.196,2.118)}
\gppoint{gp mark 2}{(3.968,2.161)}
\gppoint{gp mark 2}{(4.740,2.282)}
\gppoint{gp mark 2}{(5.513,2.460)}
\gppoint{gp mark 2}{(6.285,2.629)}
\gppoint{gp mark 2}{(7.057,2.813)}
\gppoint{gp mark 2}{(7.829,2.999)}
\gppoint{gp mark 2}{(4.484,3.930)}
\gpcolor{color=gp lt color border}
\node[gp node right] at (6.598,4.238) {MKL};
\gpcolor{color=gp lt color 2}
\gpsetlinetype{gp lt plot 2}
\draw[gp path] (6.782,4.238)--(7.698,4.238);
\draw[gp path] (2.424,1.497)--(3.196,1.523)--(3.968,1.714)--(4.740,1.842)--(5.513,2.030)%
  --(6.285,2.244)--(7.057,2.638)--(7.829,2.748);
\gppoint{gp mark 3}{(2.424,1.497)}
\gppoint{gp mark 3}{(3.196,1.523)}
\gppoint{gp mark 3}{(3.968,1.714)}
\gppoint{gp mark 3}{(4.740,1.842)}
\gppoint{gp mark 3}{(5.513,2.030)}
\gppoint{gp mark 3}{(6.285,2.244)}
\gppoint{gp mark 3}{(7.057,2.638)}
\gppoint{gp mark 3}{(7.829,2.748)}
\gppoint{gp mark 3}{(7.240,4.238)}
\gpcolor{color=gp lt color border}
\node[gp node right] at (6.598,3.930) {Trilinos};
\gpcolor{color=gp lt color 3}
\gpsetlinetype{gp lt plot 3}
\draw[gp path] (6.782,3.930)--(7.698,3.930);
\draw[gp path] (2.424,1.314)--(3.196,1.395)--(3.968,1.648)--(4.740,1.887)--(5.513,1.990)%
  --(6.285,2.304)--(7.057,2.533)--(7.829,2.722);
\gppoint{gp mark 4}{(2.424,1.314)}
\gppoint{gp mark 4}{(3.196,1.395)}
\gppoint{gp mark 4}{(3.968,1.648)}
\gppoint{gp mark 4}{(4.740,1.887)}
\gppoint{gp mark 4}{(5.513,1.990)}
\gppoint{gp mark 4}{(6.285,2.304)}
\gppoint{gp mark 4}{(7.057,2.533)}
\gppoint{gp mark 4}{(7.829,2.722)}
\gppoint{gp mark 4}{(7.240,3.930)}
\gpcolor{color=gp lt color border}
\gpsetlinetype{gp lt border}
\draw[gp path] (2.424,3.279)--(2.424,0.985)--(7.829,0.985)--(7.829,3.279)--cycle;
%% coordinates of the plot area
\gpdefrectangularnode{gp plot 1}{\pgfpoint{2.424cm}{0.985cm}}{\pgfpoint{7.829cm}{3.279cm}}
\end{tikzpicture}
%% gnuplot variables

		\vspace{-15pt}
		\caption{The runtime of dense matrix multiplication in $op1$. We compare
			in-memory (FE-IM) and external-memory (FE-EM) implementations
			in FlashEigen with the MKL and Trilinos implementations.}
		\label{perf:gemm}
	\end{center}
\end{figure}

Figure \ref{perf:gemm} show the performance of the in-memory and external-memory
dense matrix multiplication in the first form. We omit the performance result
of dense matrix multiplication in the second form because it is very similar
to the first form and MKL cannot parallelize it.
The external-memory multiplication in FlashEigen is roughly 3-6 times slower
than its in-memory counterpart. In the dense matrix multiplication of
the first form, the in-memory implementation outperforms MKL and Trilinos
when the number of the columns get larger and MKL does not have a parallel
implementation for the dense matrix multiplication of the second form.

\begin{figure}
	\begin{center}
		\footnotesize
		\vspace{-15pt}
		\begin{tikzpicture}[gnuplot]
%% generated with GNUPLOT 4.6p4 (Lua 5.1; terminal rev. 99, script rev. 100)
%% Sat 03 Oct 2015 04:36:00 PM EDT
\path (0.000,0.000) rectangle (8.382,4.572);
\gpcolor{color=gp lt color border}
\gpsetlinetype{gp lt border}
\gpsetlinewidth{1.00}
\draw[gp path] (1.320,0.985)--(1.500,0.985);
\draw[gp path] (7.829,0.985)--(7.649,0.985);
\node[gp node right] at (1.136,0.985) { 0};
\draw[gp path] (1.320,1.419)--(1.500,1.419);
\draw[gp path] (7.829,1.419)--(7.649,1.419);
\node[gp node right] at (1.136,1.419) { 2};
\draw[gp path] (1.320,1.852)--(1.500,1.852);
\draw[gp path] (7.829,1.852)--(7.649,1.852);
\node[gp node right] at (1.136,1.852) { 4};
\draw[gp path] (1.320,2.286)--(1.500,2.286);
\draw[gp path] (7.829,2.286)--(7.649,2.286);
\node[gp node right] at (1.136,2.286) { 6};
\draw[gp path] (1.320,2.720)--(1.500,2.720);
\draw[gp path] (7.829,2.720)--(7.649,2.720);
\node[gp node right] at (1.136,2.720) { 8};
\draw[gp path] (1.320,3.153)--(1.500,3.153);
\draw[gp path] (7.829,3.153)--(7.649,3.153);
\node[gp node right] at (1.136,3.153) { 10};
\draw[gp path] (1.320,3.587)--(1.500,3.587);
\draw[gp path] (7.829,3.587)--(7.649,3.587);
\node[gp node right] at (1.136,3.587) { 12};
\draw[gp path] (1.320,0.985)--(1.320,1.165);
\draw[gp path] (1.320,3.587)--(1.320,3.407);
\node[gp node center] at (1.320,0.677) {4};
\draw[gp path] (2.250,0.985)--(2.250,1.165);
\draw[gp path] (2.250,3.587)--(2.250,3.407);
\node[gp node center] at (2.250,0.677) {8};
\draw[gp path] (3.180,0.985)--(3.180,1.165);
\draw[gp path] (3.180,3.587)--(3.180,3.407);
\node[gp node center] at (3.180,0.677) {16};
\draw[gp path] (4.110,0.985)--(4.110,1.165);
\draw[gp path] (4.110,3.587)--(4.110,3.407);
\node[gp node center] at (4.110,0.677) {32};
\draw[gp path] (5.039,0.985)--(5.039,1.165);
\draw[gp path] (5.039,3.587)--(5.039,3.407);
\node[gp node center] at (5.039,0.677) {64};
\draw[gp path] (5.969,0.985)--(5.969,1.165);
\draw[gp path] (5.969,3.587)--(5.969,3.407);
\node[gp node center] at (5.969,0.677) {128};
\draw[gp path] (6.899,0.985)--(6.899,1.165);
\draw[gp path] (6.899,3.587)--(6.899,3.407);
\node[gp node center] at (6.899,0.677) {256};
\draw[gp path] (7.829,0.985)--(7.829,1.165);
\draw[gp path] (7.829,3.587)--(7.829,3.407);
\node[gp node center] at (7.829,0.677) {512};
\draw[gp path] (1.320,3.587)--(1.320,0.985)--(7.829,0.985)--(7.829,3.587)--cycle;
\node[gp node center,rotate=-270] at (0.246,2.286) {GB/s};
\node[gp node center] at (4.574,0.215) {The number of columns};
\node[gp node right] at (3.290,4.238) {op1};
\gpcolor{color=gp lt color 0}
\gpsetlinetype{gp lt plot 0}
\draw[gp path] (3.474,4.238)--(4.390,4.238);
\draw[gp path] (1.320,2.191)--(2.250,2.275)--(3.180,2.819)--(4.110,3.110)--(5.039,3.086)%
  --(5.969,3.199)--(6.899,3.225)--(7.829,3.249);
\gpsetpointsize{4.00}
\gppoint{gp mark 1}{(1.320,2.191)}
\gppoint{gp mark 1}{(2.250,2.275)}
\gppoint{gp mark 1}{(3.180,2.819)}
\gppoint{gp mark 1}{(4.110,3.110)}
\gppoint{gp mark 1}{(5.039,3.086)}
\gppoint{gp mark 1}{(5.969,3.199)}
\gppoint{gp mark 1}{(6.899,3.225)}
\gppoint{gp mark 1}{(7.829,3.249)}
\gppoint{gp mark 1}{(3.932,4.238)}
\gpcolor{color=gp lt color border}
\node[gp node right] at (5.126,4.238) {op3};
\gpcolor{color=gp lt color 1}
\gpsetlinetype{gp lt plot 1}
\draw[gp path] (5.310,4.238)--(6.226,4.238);
\draw[gp path] (1.320,2.444)--(2.250,2.902)--(3.180,2.921)--(4.110,2.913)--(5.039,3.169)%
  --(5.969,3.299)--(6.899,3.320)--(7.829,3.342);
\gppoint{gp mark 2}{(1.320,2.444)}
\gppoint{gp mark 2}{(2.250,2.902)}
\gppoint{gp mark 2}{(3.180,2.921)}
\gppoint{gp mark 2}{(4.110,2.913)}
\gppoint{gp mark 2}{(5.039,3.169)}
\gppoint{gp mark 2}{(5.969,3.299)}
\gppoint{gp mark 2}{(6.899,3.320)}
\gppoint{gp mark 2}{(7.829,3.342)}
\gppoint{gp mark 2}{(5.768,4.238)}
\gpcolor{color=gp lt color border}
\gpsetlinetype{gp lt border}
\draw[gp path] (1.320,3.587)--(1.320,0.985)--(7.829,0.985)--(7.829,3.587)--cycle;
%% coordinates of the plot area
\gpdefrectangularnode{gp plot 1}{\pgfpoint{1.320cm}{0.985cm}}{\pgfpoint{7.829cm}{3.587cm}}
\end{tikzpicture}
%% gnuplot variables

		\vspace{-15pt}
		\caption{The average I/O throughput in dense matrix multiplication
		on the SSD array.}
		\label{perf:dmm_io}
	\end{center}
\end{figure}

The external-memory dense matrix multiplication has almost saturated
the I/O bandwidth of the SSDs (Figure \ref{perf:dmm_io}). The average
I/O throughput reaches 10.87GB/s from the entire SSD array or 464MB/s
per SSD, while the peak throughput of an SSD is approaching 500MB/s,
which is close to the maximal I/O throughput of 540MB/s per SSD, advertised by
the SSD vendor. This also indicates that the SSDs are the bottleneck for
the dense matrix multiplication in the eigensolver. This is not
surprising because the sequential I/O performance of SSDs is an order of
magnitude smaller than RAM. 

\subsection{The KrylovSchur eigensolver in FlashEigen}
In this section, we focus on evaluating the performance of the KrylovSchur
eigensolvers in FlashEigen. KrylovSchur is not only the fastest in-memory
eigensolver on the graphs in Table \ref{graphs} among all Anasazi eigensolvers,
but also generates the least I/O, especially writes,
to SSDs. Reducing I/O is essential to achieve performance and reduce SSD
wearout. We evaluate our external-memory eigensolver and compare it
with its in-memory counterpart as well as the original KrylovSchur eigensolver
in the Anasazi framework.

The KrylovSchur eigensolver has two important parameters: the subspace size
and the block size. They significantly affect the convergence of
the KrylovSchur eigensolver. A reasonably large subspace size and block size
accelerates its convergence. However, a larger subspace increases memory
consumption of the eigensolver as well as computation and I/O complexity
in a single matrix operation. A larger block size increases memory consumption
of FlashEigen.

For the experiments below, we select the values for the two parameters that
achieve the best runtime performance for the eigensolvers. We assume that
our setting is not constrained by memory size available to the machine.
Because sparse matrix in Trilinos is not optimized for
the dense matrix with more than one column, we use $1$ as the block size and
$2 \times ev$ as the number of blocks for most of the graphs in the original
KrylovSchur eigensolver, where $ev$ is the number of the eigenvalues to compute.
For the FlashEigen KrylovSchur eigensolver, we choose the subspace size and
the block size to reduce the amount of I/O to SSDs. we use $1$ as the block size and
$2 \times ev$ as the number of blocks when computing a small number of eigenvalues
and use $4$ as the block size and $ev$ as the number of blocks. Therefore,
the FlashEigen eigensolver uses the subspace twice as large as the one used by
the original KrylovSchur eigensolver when computing a large number of eigenvalues.
However, the eigenvalues of the graph W are very close to each other, so we have
to use a much larger subspace to have the eigensolver to converge or converge
faster. For the W graph, we use the subspace four times larger than the ones
used for other graphs. Therefore, we can trade off the runtime performance
with memory consumption in eigendecomposition.

\subsubsection{Performance of eigensolvers}

We evaluate the performance of our SEM eigensolver and compare its performance
with our in-memory eigensolver and the original KrylovSchur eigensolver.
We compute different numbers of eigenvalues on the three smaller graphs in
Table \ref{graphs}. Only our SEM eigensolver is able to compute eigenvalues
on the page graph on the 1TB-memory machine.

\begin{figure}[t]
\centering
\footnotesize
\vspace{-15pt}
\begin{subfigure}{.5\linewidth}
	\begin{tikzpicture}[gnuplot]
%% generated with GNUPLOT 4.6p4 (Lua 5.1; terminal rev. 99, script rev. 100)
%% Wed 11 Nov 2015 10:12:11 AM EST
\path (0.000,0.000) rectangle (5.080,4.572);
\gpcolor{color=gp lt color border}
\gpsetlinetype{gp lt border}
\gpsetlinewidth{1.00}
\draw[gp path] (1.196,1.126)--(1.376,1.126);
\draw[gp path] (4.527,1.126)--(4.347,1.126);
\node[gp node right] at (1.012,1.126) { 0};
\draw[gp path] (1.196,1.557)--(1.376,1.557);
\draw[gp path] (4.527,1.557)--(4.347,1.557);
\node[gp node right] at (1.012,1.557) { 0.2};
\draw[gp path] (1.196,1.987)--(1.376,1.987);
\draw[gp path] (4.527,1.987)--(4.347,1.987);
\node[gp node right] at (1.012,1.987) { 0.4};
\draw[gp path] (1.196,2.418)--(1.376,2.418);
\draw[gp path] (4.527,2.418)--(4.347,2.418);
\node[gp node right] at (1.012,2.418) { 0.6};
\draw[gp path] (1.196,2.848)--(1.376,2.848);
\draw[gp path] (4.527,2.848)--(4.347,2.848);
\node[gp node right] at (1.012,2.848) { 0.8};
\draw[gp path] (1.196,3.279)--(1.376,3.279);
\draw[gp path] (4.527,3.279)--(4.347,3.279);
\node[gp node right] at (1.012,3.279) { 1};
\draw[gp path] (1.862,1.126)--(1.862,1.306);
\draw[gp path] (1.862,3.279)--(1.862,3.099);
\node[gp node left,rotate=-20] at (1.862,0.942) {rmat-100M-40};
\draw[gp path] (2.528,1.126)--(2.528,1.306);
\draw[gp path] (2.528,3.279)--(2.528,3.099);
\node[gp node left,rotate=-20] at (2.528,0.942) {rmat-100M-160};
\draw[gp path] (3.195,1.126)--(3.195,1.306);
\draw[gp path] (3.195,3.279)--(3.195,3.099);
\node[gp node left,rotate=-20] at (3.195,0.942) {Friendster};
\draw[gp path] (3.861,1.126)--(3.861,1.306);
\draw[gp path] (3.861,3.279)--(3.861,3.099);
\node[gp node left,rotate=-20] at (3.861,0.942) {W};
\draw[gp path] (1.196,3.279)--(1.196,1.126)--(4.527,1.126)--(4.527,3.279)--cycle;
\node[gp node right] at (2.955,4.238) {Trilinos};
\def\gpfillpath{(3.139,4.161)--(4.055,4.161)--(4.055,4.315)--(3.139,4.315)--cycle}
\gpfill{color=gpbgfillcolor} \gpfillpath;
\gpfill{color=gp lt color 0,gp pattern 0,pattern color=.} \gpfillpath;
\gpcolor{color=gp lt color 0}
\gpsetlinetype{gp lt plot 0}
\draw[gp path] (3.139,4.161)--(4.055,4.161)--(4.055,4.315)--(3.139,4.315)--cycle;
\def\gpfillpath{(1.779,1.126)--(1.946,1.126)--(1.946,2.333)--(1.779,2.333)--cycle}
\gpfill{color=gpbgfillcolor} \gpfillpath;
\gpfill{color=gp lt color 0,gp pattern 0,pattern color=.} \gpfillpath;
\draw[gp path] (1.779,1.126)--(1.779,2.332)--(1.945,2.332)--(1.945,1.126)--cycle;
\def\gpfillpath{(2.445,1.126)--(2.613,1.126)--(2.613,2.354)--(2.445,2.354)--cycle}
\gpfill{color=gpbgfillcolor} \gpfillpath;
\gpfill{color=gp lt color 0,gp pattern 0,pattern color=.} \gpfillpath;
\draw[gp path] (2.445,1.126)--(2.445,2.353)--(2.612,2.353)--(2.612,1.126)--cycle;
\def\gpfillpath{(3.111,1.126)--(3.279,1.126)--(3.279,2.397)--(3.111,2.397)--cycle}
\gpfill{color=gpbgfillcolor} \gpfillpath;
\gpfill{color=gp lt color 0,gp pattern 0,pattern color=.} \gpfillpath;
\draw[gp path] (3.111,1.126)--(3.111,2.396)--(3.278,2.396)--(3.278,1.126)--cycle;
\def\gpfillpath{(3.778,1.126)--(3.945,1.126)--(3.945,2.613)--(3.778,2.613)--cycle}
\gpfill{color=gpbgfillcolor} \gpfillpath;
\gpfill{color=gp lt color 0,gp pattern 0,pattern color=.} \gpfillpath;
\draw[gp path] (3.778,1.126)--(3.778,2.612)--(3.944,2.612)--(3.944,1.126)--cycle;
\gpcolor{color=gp lt color border}
\node[gp node right] at (2.955,3.930) {EM};
\def\gpfillpath{(3.139,3.853)--(4.055,3.853)--(4.055,4.007)--(3.139,4.007)--cycle}
\gpfill{color=gpbgfillcolor} \gpfillpath;
\gpfill{color=gp lt color 1,gp pattern 1,pattern color=.} \gpfillpath;
\gpcolor{color=gp lt color 1}
\gpsetlinetype{gp lt plot 1}
\draw[gp path] (3.139,3.853)--(4.055,3.853)--(4.055,4.007)--(3.139,4.007)--cycle;
\def\gpfillpath{(1.945,1.126)--(2.113,1.126)--(2.113,2.247)--(1.945,2.247)--cycle}
\gpfill{color=gpbgfillcolor} \gpfillpath;
\gpfill{color=gp lt color 1,gp pattern 1,pattern color=.} \gpfillpath;
\draw[gp path] (1.945,1.126)--(1.945,2.246)--(2.112,2.246)--(2.112,1.126)--cycle;
\def\gpfillpath{(2.612,1.126)--(2.779,1.126)--(2.779,2.268)--(2.612,2.268)--cycle}
\gpfill{color=gpbgfillcolor} \gpfillpath;
\gpfill{color=gp lt color 1,gp pattern 1,pattern color=.} \gpfillpath;
\draw[gp path] (2.612,1.126)--(2.612,2.267)--(2.778,2.267)--(2.778,1.126)--cycle;
\def\gpfillpath{(3.278,1.126)--(3.445,1.126)--(3.445,1.967)--(3.278,1.967)--cycle}
\gpfill{color=gpbgfillcolor} \gpfillpath;
\gpfill{color=gp lt color 1,gp pattern 1,pattern color=.} \gpfillpath;
\draw[gp path] (3.278,1.126)--(3.278,1.966)--(3.444,1.966)--(3.444,1.126)--cycle;
\def\gpfillpath{(3.944,1.126)--(4.112,1.126)--(4.112,2.440)--(3.944,2.440)--cycle}
\gpfill{color=gpbgfillcolor} \gpfillpath;
\gpfill{color=gp lt color 1,gp pattern 1,pattern color=.} \gpfillpath;
\draw[gp path] (3.944,1.126)--(3.944,2.439)--(4.111,2.439)--(4.111,1.126)--cycle;
\gpcolor{color=gp lt color border}
\gpsetlinetype{gp lt border}
\draw[gp path] (1.196,3.279)--(1.196,1.126)--(4.527,1.126)--(4.527,3.279)--cycle;
%% coordinates of the plot area
\gpdefrectangularnode{gp plot 1}{\pgfpoint{1.196cm}{1.126cm}}{\pgfpoint{4.527cm}{3.279cm}}
\end{tikzpicture}
%% gnuplot variables

	\vspace{-15pt}
	\caption{8 eigenvalues.}
	\label{fig:eigen8}
\end{subfigure}%
\begin{subfigure}{.5\linewidth}
	\begin{tikzpicture}[gnuplot]
%% generated with GNUPLOT 4.6p4 (Lua 5.1; terminal rev. 99, script rev. 100)
%% Wed 11 Nov 2015 10:12:14 AM EST
\path (0.000,0.000) rectangle (5.080,4.572);
\gpcolor{color=gp lt color border}
\gpsetlinetype{gp lt border}
\gpsetlinewidth{1.00}
\draw[gp path] (1.196,1.126)--(1.376,1.126);
\draw[gp path] (4.527,1.126)--(4.347,1.126);
\node[gp node right] at (1.012,1.126) { 0};
\draw[gp path] (1.196,1.557)--(1.376,1.557);
\draw[gp path] (4.527,1.557)--(4.347,1.557);
\node[gp node right] at (1.012,1.557) { 0.2};
\draw[gp path] (1.196,1.987)--(1.376,1.987);
\draw[gp path] (4.527,1.987)--(4.347,1.987);
\node[gp node right] at (1.012,1.987) { 0.4};
\draw[gp path] (1.196,2.418)--(1.376,2.418);
\draw[gp path] (4.527,2.418)--(4.347,2.418);
\node[gp node right] at (1.012,2.418) { 0.6};
\draw[gp path] (1.196,2.848)--(1.376,2.848);
\draw[gp path] (4.527,2.848)--(4.347,2.848);
\node[gp node right] at (1.012,2.848) { 0.8};
\draw[gp path] (1.196,3.279)--(1.376,3.279);
\draw[gp path] (4.527,3.279)--(4.347,3.279);
\node[gp node right] at (1.012,3.279) { 1};
\draw[gp path] (1.862,1.126)--(1.862,1.306);
\draw[gp path] (1.862,3.279)--(1.862,3.099);
\node[gp node left,rotate=-20] at (1.862,0.942) {rmat-100M-40};
\draw[gp path] (2.528,1.126)--(2.528,1.306);
\draw[gp path] (2.528,3.279)--(2.528,3.099);
\node[gp node left,rotate=-20] at (2.528,0.942) {rmat-100M-160};
\draw[gp path] (3.195,1.126)--(3.195,1.306);
\draw[gp path] (3.195,3.279)--(3.195,3.099);
\node[gp node left,rotate=-20] at (3.195,0.942) {Friendster};
\draw[gp path] (3.861,1.126)--(3.861,1.306);
\draw[gp path] (3.861,3.279)--(3.861,3.099);
\node[gp node left,rotate=-20] at (3.861,0.942) {W};
\draw[gp path] (1.196,3.279)--(1.196,1.126)--(4.527,1.126)--(4.527,3.279)--cycle;
\node[gp node right] at (2.955,4.238) {Trilinos};
\def\gpfillpath{(3.139,4.161)--(4.055,4.161)--(4.055,4.315)--(3.139,4.315)--cycle}
\gpfill{color=gpbgfillcolor} \gpfillpath;
\gpfill{color=gp lt color 0,gp pattern 0,pattern color=.} \gpfillpath;
\gpcolor{color=gp lt color 0}
\gpsetlinetype{gp lt plot 0}
\draw[gp path] (3.139,4.161)--(4.055,4.161)--(4.055,4.315)--(3.139,4.315)--cycle;
\def\gpfillpath{(1.779,1.126)--(1.946,1.126)--(1.946,2.634)--(1.779,2.634)--cycle}
\gpfill{color=gpbgfillcolor} \gpfillpath;
\gpfill{color=gp lt color 0,gp pattern 0,pattern color=.} \gpfillpath;
\draw[gp path] (1.779,1.126)--(1.779,2.633)--(1.945,2.633)--(1.945,1.126)--cycle;
\def\gpfillpath{(2.445,1.126)--(2.613,1.126)--(2.613,2.526)--(2.445,2.526)--cycle}
\gpfill{color=gpbgfillcolor} \gpfillpath;
\gpfill{color=gp lt color 0,gp pattern 0,pattern color=.} \gpfillpath;
\draw[gp path] (2.445,1.126)--(2.445,2.525)--(2.612,2.525)--(2.612,1.126)--cycle;
\def\gpfillpath{(3.111,1.126)--(3.279,1.126)--(3.279,3.280)--(3.111,3.280)--cycle}
\gpfill{color=gpbgfillcolor} \gpfillpath;
\gpfill{color=gp lt color 0,gp pattern 0,pattern color=.} \gpfillpath;
\draw[gp path] (3.111,1.126)--(3.111,3.279)--(3.278,3.279)--(3.278,1.126)--cycle;
\def\gpfillpath{(3.778,1.126)--(3.945,1.126)--(3.945,3.000)--(3.778,3.000)--cycle}
\gpfill{color=gpbgfillcolor} \gpfillpath;
\gpfill{color=gp lt color 0,gp pattern 0,pattern color=.} \gpfillpath;
\draw[gp path] (3.778,1.126)--(3.778,2.999)--(3.944,2.999)--(3.944,1.126)--cycle;
\gpcolor{color=gp lt color border}
\node[gp node right] at (2.955,3.930) {EM};
\def\gpfillpath{(3.139,3.853)--(4.055,3.853)--(4.055,4.007)--(3.139,4.007)--cycle}
\gpfill{color=gpbgfillcolor} \gpfillpath;
\gpfill{color=gp lt color 1,gp pattern 1,pattern color=.} \gpfillpath;
\gpcolor{color=gp lt color 1}
\gpsetlinetype{gp lt plot 1}
\draw[gp path] (3.139,3.853)--(4.055,3.853)--(4.055,4.007)--(3.139,4.007)--cycle;
\def\gpfillpath{(1.945,1.126)--(2.113,1.126)--(2.113,1.988)--(1.945,1.988)--cycle}
\gpfill{color=gpbgfillcolor} \gpfillpath;
\gpfill{color=gp lt color 1,gp pattern 1,pattern color=.} \gpfillpath;
\draw[gp path] (1.945,1.126)--(1.945,1.987)--(2.112,1.987)--(2.112,1.126)--cycle;
\def\gpfillpath{(2.612,1.126)--(2.779,1.126)--(2.779,2.462)--(2.612,2.462)--cycle}
\gpfill{color=gpbgfillcolor} \gpfillpath;
\gpfill{color=gp lt color 1,gp pattern 1,pattern color=.} \gpfillpath;
\draw[gp path] (2.612,1.126)--(2.612,2.461)--(2.778,2.461)--(2.778,1.126)--cycle;
\def\gpfillpath{(3.278,1.126)--(3.445,1.126)--(3.445,1.967)--(3.278,1.967)--cycle}
\gpfill{color=gpbgfillcolor} \gpfillpath;
\gpfill{color=gp lt color 1,gp pattern 1,pattern color=.} \gpfillpath;
\draw[gp path] (3.278,1.126)--(3.278,1.966)--(3.444,1.966)--(3.444,1.126)--cycle;
\def\gpfillpath{(3.944,1.126)--(4.112,1.126)--(4.112,2.160)--(3.944,2.160)--cycle}
\gpfill{color=gpbgfillcolor} \gpfillpath;
\gpfill{color=gp lt color 1,gp pattern 1,pattern color=.} \gpfillpath;
\draw[gp path] (3.944,1.126)--(3.944,2.159)--(4.111,2.159)--(4.111,1.126)--cycle;
\gpcolor{color=gp lt color border}
\gpsetlinetype{gp lt border}
\draw[gp path] (1.196,3.279)--(1.196,1.126)--(4.527,1.126)--(4.527,3.279)--cycle;
%% coordinates of the plot area
\gpdefrectangularnode{gp plot 1}{\pgfpoint{1.196cm}{1.126cm}}{\pgfpoint{4.527cm}{3.279cm}}
\end{tikzpicture}
%% gnuplot variables

	\vspace{-15pt}
	\caption{32 eigenvalues.}
	\label{fig:eigen32}
\end{subfigure}
\caption{The performance of the Trilinos KrylovSchur and FlashEigen-EM
KrylovSchur relative to the FlashEigen-IM KrylovSchur.}
\vspace{-15pt}
\label{fig:eigen}
\end{figure}

Our SEM eigensolver achieves at least 40\% performance of our in-memory
eigensolver, while the in-memory eigensolver outperforms the original
KrylovSchur eigensolver (Figure \ref{fig:eigen}). The SEM eigensolver
is more efficient to compute a small number of eigenvalues and is able
to achieve around 50\% performance of our in-memory eigensolver.
SpMM and reorthogonalization
account for most of computation time when computing a small number of eigenvalues,
but reorthogonalization eventually dominates the eigendecomposition for computing
many eigenvalues. Because external-memory dense matrix multiplication is several
times slower than the in-memory implementations, reorthogonalization accounts for
over 90\% of runtime in the SEM eigensolver for computing a large number of
eigenvalues. For many spectral analysis tasks, users only require a small number
of eigenvalues.

The SEM eigensolver uses a small fraction of memory used by its in-memory
counterparts and the original Trilinos eigensolver and its memory consumption
remains roughly the same as the number of eigenvalues computed by the eigensolvers
increases. A small memory consumption provides two benefits. First, FlashEigen
is able to scale to a much larger eigenvalue problem. The second benefit is that
FlashEigen gives users more freedom to choose the subspace size that gives the
fastest convergence in a given eigenvalue problem because SSDs significantly
increases memory available to the eigensolver. We have to point out that our
experiments assume that we choose the subspace size that is not constrained by
the memory size in a machine.

%\begin{figure}
%	\begin{center}
%		\footnotesize
%		\vspace{-15pt}
%		\include{diff_cache}
%		\vspace{-15pt}
%		\caption{The performance impact of caching more vectors in the subspace
%		in FlashEigen}
%		\label{perf:cache}
%	\end{center}
%\end{figure}

\subsubsection{Scale to billion-node graphs}

We evaluate the scalability of FlashEigen with the page graph with 3.4 billion
vertices and 129 billion edges. Because the page graph is a directed graph,
its adjacency matrix is asymmetric and we perform singular value decomposition
(SVD) on the adjacency matrix instead of simple eigendecomposition. For the page
graph, we use $2$ for the block size and $2 \times ev$ for the number of blocks
because sparse matrix multiplication is completely bottlenecked by SSDs.
Neither the in-memory eigensolver nor the original Trilinos eigensolver is able
to compute eigenvalues on the page graph with 1TB RAM.

\begin{table}
	\begin{center}
		\small
		\begin{tabular}{|c|c|c|c|c|}
			\hline
			\#eigenvalues & runtime & memory & read & write \\
			\hline
			8 & 4.2 hours & 120GB & 145TB & 4TB \\
			\hline
		\end{tabular}
		\normalsize
	\end{center}
	\caption{The performance and resource consumption of computing eigenvalues
	on the page graph.}
	\label{pg_ev}
\end{table}

FlashEigen computes a fairly large number of eigenvalues within a reasonable
amount of time and consumes a fairly small amount of resources given the large
size of the eigenvalue problem.
The average I/O throughput during the computation of 8 eigenvalues is about
10GB/s, which is very close to the maximal I/O throughput provided by
the SSD array.
Given the relative small memory footprint, we are able to scale FlashEigen
to a much larger eigenvalue problem on our 1TB-RAM machine.
