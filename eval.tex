\section{Experimental Evaluation}

We use a very large SAFS RAID block size. We should mention the sequential I/O performance
of the SSD array.

\subsection{Matrix multiplication}

Matrix multiplication is the most important operation in an eigensolver.
Both sparse matrix multiplication and dense matrix multiplication together
account for most of the computation.

\subsubsection{Sparse matrix dense matrix multiplication}
This section evaluates the performance of the in-memory (mem), semi-external-memory
(SEM) and external-memory (EM) version of SpMM in our eigensolvers. We compares
their performance with that of Intel MKL implementation.
We should vary the number of columns in the dense matrix to show that EM SpMM and SEM SpMM will
eventually catch the performance of in-mem SpMM. Another point is that our SpMM outperforms
Intel MKL.

\dz{I should also plot a chart to show the effectiveness of each optimization
in SpMM.}

\begin{figure}
	\begin{center}
		\footnotesize
		\vspace{-15pt}
		\begin{tikzpicture}[gnuplot]
%% generated with GNUPLOT 4.6p4 (Lua 5.1; terminal rev. 99, script rev. 100)
%% Wed 22 Jul 2015 03:40:43 PM EDT
\path (0.000,0.000) rectangle (8.382,4.572);
\gpcolor{color=gp lt color border}
\gpsetlinetype{gp lt border}
\gpsetlinewidth{1.00}
\draw[gp path] (1.504,0.985)--(1.684,0.985);
\draw[gp path] (7.829,0.985)--(7.649,0.985);
\node[gp node right] at (1.320,0.985) { 0};
\draw[gp path] (1.504,1.357)--(1.684,1.357);
\draw[gp path] (7.829,1.357)--(7.649,1.357);
\node[gp node right] at (1.320,1.357) { 0.5};
\draw[gp path] (1.504,1.728)--(1.684,1.728);
\draw[gp path] (7.829,1.728)--(7.649,1.728);
\node[gp node right] at (1.320,1.728) { 1};
\draw[gp path] (1.504,2.100)--(1.684,2.100);
\draw[gp path] (7.829,2.100)--(7.649,2.100);
\node[gp node right] at (1.320,2.100) { 1.5};
\draw[gp path] (1.504,2.472)--(1.684,2.472);
\draw[gp path] (7.829,2.472)--(7.649,2.472);
\node[gp node right] at (1.320,2.472) { 2};
\draw[gp path] (1.504,2.844)--(1.684,2.844);
\draw[gp path] (7.829,2.844)--(7.649,2.844);
\node[gp node right] at (1.320,2.844) { 2.5};
\draw[gp path] (1.504,3.215)--(1.684,3.215);
\draw[gp path] (7.829,3.215)--(7.649,3.215);
\node[gp node right] at (1.320,3.215) { 3};
\draw[gp path] (1.504,3.587)--(1.684,3.587);
\draw[gp path] (7.829,3.587)--(7.649,3.587);
\node[gp node right] at (1.320,3.587) { 3.5};
\draw[gp path] (1.504,0.985)--(1.504,1.165);
\draw[gp path] (1.504,3.587)--(1.504,3.407);
\node[gp node center] at (1.504,0.677) {1};
\draw[gp path] (3.085,0.985)--(3.085,1.165);
\draw[gp path] (3.085,3.587)--(3.085,3.407);
\node[gp node center] at (3.085,0.677) {2};
\draw[gp path] (4.667,0.985)--(4.667,1.165);
\draw[gp path] (4.667,3.587)--(4.667,3.407);
\node[gp node center] at (4.667,0.677) {4};
\draw[gp path] (6.248,0.985)--(6.248,1.165);
\draw[gp path] (6.248,3.587)--(6.248,3.407);
\node[gp node center] at (6.248,0.677) {8};
\draw[gp path] (7.829,0.985)--(7.829,1.165);
\draw[gp path] (7.829,3.587)--(7.829,3.407);
\node[gp node center] at (7.829,0.677) {16};
\draw[gp path] (1.504,3.587)--(1.504,0.985)--(7.829,0.985)--(7.829,3.587)--cycle;
\node[gp node center,rotate=-270] at (0.246,2.286) {Relative performance};
\node[gp node center] at (4.666,0.215) {The number of columns of the dense matrix};
\node[gp node right] at (3.382,4.238) {mem};
\gpcolor{color=gp lt color 0}
\gpsetlinetype{gp lt plot 0}
\draw[gp path] (3.566,4.238)--(4.482,4.238);
\draw[gp path] (1.504,3.327)--(3.085,2.903)--(4.667,2.606)--(6.248,2.338)--(7.829,2.160);
\gpsetpointsize{4.00}
\gppoint{gp mark 1}{(1.504,3.327)}
\gppoint{gp mark 1}{(3.085,2.903)}
\gppoint{gp mark 1}{(4.667,2.606)}
\gppoint{gp mark 1}{(6.248,2.338)}
\gppoint{gp mark 1}{(7.829,2.160)}
\gppoint{gp mark 1}{(4.024,4.238)}
\gpcolor{color=gp lt color border}
\node[gp node right] at (5.218,4.238) {SEM};
\gpcolor{color=gp lt color 1}
\gpsetlinetype{gp lt plot 1}
\draw[gp path] (5.402,4.238)--(6.318,4.238);
\draw[gp path] (1.504,0.985)--(3.085,0.985)--(4.667,0.985)--(6.248,0.985)--(7.829,0.985);
\gppoint{gp mark 2}{(1.504,0.985)}
\gppoint{gp mark 2}{(3.085,0.985)}
\gppoint{gp mark 2}{(4.667,0.985)}
\gppoint{gp mark 2}{(6.248,0.985)}
\gppoint{gp mark 2}{(7.829,0.985)}
\gppoint{gp mark 2}{(5.860,4.238)}
\gpcolor{color=gp lt color border}
\gpsetlinetype{gp lt border}
\draw[gp path] (1.504,3.587)--(1.504,0.985)--(7.829,0.985)--(7.829,3.587)--cycle;
%% coordinates of the plot area
\gpdefrectangularnode{gp plot 1}{\pgfpoint{1.504cm}{0.985cm}}{\pgfpoint{7.829cm}{3.587cm}}
\end{tikzpicture}
%% gnuplot variables

		\vspace{-15pt}
		\caption{The performance of the in-memory, SEM and EM implementations
			of SpMM in our eigensolvers on the Friendster graph. The performance
			is normalized to the Intel MKL. \dz{measure the SEM performance.}}
		\label{perf:spmm}
	\end{center}
\end{figure}

One of the reason that MKL is slow is that MKL cannot balance the load well.

\begin{figure}
	\begin{center}
		\footnotesize
		\vspace{-15pt}
		\begin{tikzpicture}[gnuplot]
%% generated with GNUPLOT 4.6p4 (Lua 5.1; terminal rev. 99, script rev. 100)
%% Wed 22 Jul 2015 12:34:54 AM EDT
\path (0.000,0.000) rectangle (8.382,4.572);
\gpcolor{color=gp lt color border}
\gpsetlinetype{gp lt border}
\gpsetlinewidth{1.00}
\draw[gp path] (1.320,0.985)--(1.500,0.985);
\draw[gp path] (7.829,0.985)--(7.649,0.985);
\node[gp node right] at (1.136,0.985) { 0};
\draw[gp path] (1.320,1.505)--(1.500,1.505);
\draw[gp path] (7.829,1.505)--(7.649,1.505);
\node[gp node right] at (1.136,1.505) { 10};
\draw[gp path] (1.320,2.026)--(1.500,2.026);
\draw[gp path] (7.829,2.026)--(7.649,2.026);
\node[gp node right] at (1.136,2.026) { 20};
\draw[gp path] (1.320,2.546)--(1.500,2.546);
\draw[gp path] (7.829,2.546)--(7.649,2.546);
\node[gp node right] at (1.136,2.546) { 30};
\draw[gp path] (1.320,3.067)--(1.500,3.067);
\draw[gp path] (7.829,3.067)--(7.649,3.067);
\node[gp node right] at (1.136,3.067) { 40};
\draw[gp path] (1.320,3.587)--(1.500,3.587);
\draw[gp path] (7.829,3.587)--(7.649,3.587);
\node[gp node right] at (1.136,3.587) { 50};
\draw[gp path] (1.320,0.985)--(1.320,1.165);
\draw[gp path] (1.320,3.587)--(1.320,3.407);
\node[gp node center] at (1.320,0.677) {1};
\draw[gp path] (2.947,0.985)--(2.947,1.165);
\draw[gp path] (2.947,3.587)--(2.947,3.407);
\node[gp node center] at (2.947,0.677) {2};
\draw[gp path] (4.575,0.985)--(4.575,1.165);
\draw[gp path] (4.575,3.587)--(4.575,3.407);
\node[gp node center] at (4.575,0.677) {4};
\draw[gp path] (6.202,0.985)--(6.202,1.165);
\draw[gp path] (6.202,3.587)--(6.202,3.407);
\node[gp node center] at (6.202,0.677) {8};
\draw[gp path] (7.829,0.985)--(7.829,1.165);
\draw[gp path] (7.829,3.587)--(7.829,3.407);
\node[gp node center] at (7.829,0.677) {16};
\draw[gp path] (1.320,3.587)--(1.320,0.985)--(7.829,0.985)--(7.829,3.587)--cycle;
\node[gp node center,rotate=-270] at (0.246,2.286) {seconds};
\node[gp node center] at (4.574,0.215) {The number of columns of the dense matrix};
\node[gp node right] at (3.290,4.238) {SEM};
\gpcolor{color=gp lt color 0}
\gpsetlinetype{gp lt plot 0}
\draw[gp path] (3.474,4.238)--(4.390,4.238);
\draw[gp path] (1.320,3.587)--(2.947,3.525)--(4.575,3.509)--(6.202,3.431)--(7.829,0.985);
\gpsetpointsize{4.00}
\gppoint{gp mark 1}{(1.320,3.587)}
\gppoint{gp mark 1}{(2.947,3.525)}
\gppoint{gp mark 1}{(4.575,3.509)}
\gppoint{gp mark 1}{(6.202,3.431)}
\gppoint{gp mark 1}{(7.829,0.985)}
\gppoint{gp mark 1}{(3.932,4.238)}
\gpcolor{color=gp lt color border}
\node[gp node right] at (5.126,4.238) {mem};
\gpcolor{color=gp lt color 1}
\gpsetlinetype{gp lt plot 1}
\draw[gp path] (5.310,4.238)--(6.226,4.238);
\draw[gp path] (1.320,1.693)--(2.947,2.406)--(4.575,2.858)--(6.202,3.394)--(7.829,0.985);
\gppoint{gp mark 2}{(1.320,1.693)}
\gppoint{gp mark 2}{(2.947,2.406)}
\gppoint{gp mark 2}{(4.575,2.858)}
\gppoint{gp mark 2}{(6.202,3.394)}
\gppoint{gp mark 2}{(7.829,0.985)}
\gppoint{gp mark 2}{(5.768,4.238)}
\gpcolor{color=gp lt color border}
\gpsetlinetype{gp lt border}
\draw[gp path] (1.320,3.587)--(1.320,0.985)--(7.829,0.985)--(7.829,3.587)--cycle;
%% coordinates of the plot area
\gpdefrectangularnode{gp plot 1}{\pgfpoint{1.320cm}{0.985cm}}{\pgfpoint{7.829cm}{3.587cm}}
\end{tikzpicture}
%% gnuplot variables

		\vspace{-15pt}
		\caption{The performance of the in-memory, SEM and EM version of SpMM
			in our eigensolvers on the page graph.}
		\label{perf:spmm_pg}
	\end{center}
\end{figure}

\subsubsection{Dense matrix multiplication}

Figure \ref{perf:mat_mul} shows the relative performance of the in-memory
and external memory version of dense matrix multiplication.
There is a large gap between in-memory and external-memory implementation.
The SSDs are the bottleneck. This is not surprising because the sequential
I/O performance of SSDs is an order of magnitude smaller than RAM.
However, this indicates that we can use the extra computation to save I/O.
When the number of columns in the dense matrices increase, the performance
gap gets smaller, which indicates that when computing more eigenpairs,
the performance gap between in-memory and external-memory eigensolvers
will be smaller.

\begin{figure}
	\begin{center}
		\footnotesize
		\vspace{-15pt}
		\begin{tikzpicture}[gnuplot]
%% generated with GNUPLOT 4.6p4 (Lua 5.1; terminal rev. 99, script rev. 100)
%% Wed 22 Jul 2015 04:02:11 PM EDT
\path (0.000,0.000) rectangle (8.382,4.572);
\gpcolor{color=gp lt color border}
\gpsetlinetype{gp lt border}
\gpsetlinewidth{1.00}
\draw[gp path] (1.504,0.985)--(1.684,0.985);
\draw[gp path] (7.829,0.985)--(7.649,0.985);
\node[gp node right] at (1.320,0.985) { 0};
\draw[gp path] (1.504,1.194)--(1.684,1.194);
\draw[gp path] (7.829,1.194)--(7.649,1.194);
\node[gp node right] at (1.320,1.194) { 0.2};
\draw[gp path] (1.504,1.402)--(1.684,1.402);
\draw[gp path] (7.829,1.402)--(7.649,1.402);
\node[gp node right] at (1.320,1.402) { 0.4};
\draw[gp path] (1.504,1.611)--(1.684,1.611);
\draw[gp path] (7.829,1.611)--(7.649,1.611);
\node[gp node right] at (1.320,1.611) { 0.6};
\draw[gp path] (1.504,1.819)--(1.684,1.819);
\draw[gp path] (7.829,1.819)--(7.649,1.819);
\node[gp node right] at (1.320,1.819) { 0.8};
\draw[gp path] (1.504,2.028)--(1.684,2.028);
\draw[gp path] (7.829,2.028)--(7.649,2.028);
\node[gp node right] at (1.320,2.028) { 1};
\draw[gp path] (1.504,2.236)--(1.684,2.236);
\draw[gp path] (7.829,2.236)--(7.649,2.236);
\node[gp node right] at (1.320,2.236) { 1.2};
\draw[gp path] (1.504,2.445)--(1.684,2.445);
\draw[gp path] (7.829,2.445)--(7.649,2.445);
\node[gp node right] at (1.320,2.445) { 1.4};
\draw[gp path] (1.504,2.653)--(1.684,2.653);
\draw[gp path] (7.829,2.653)--(7.649,2.653);
\node[gp node right] at (1.320,2.653) { 1.6};
\draw[gp path] (1.504,2.862)--(1.684,2.862);
\draw[gp path] (7.829,2.862)--(7.649,2.862);
\node[gp node right] at (1.320,2.862) { 1.8};
\draw[gp path] (1.504,3.070)--(1.684,3.070);
\draw[gp path] (7.829,3.070)--(7.649,3.070);
\node[gp node right] at (1.320,3.070) { 2};
\draw[gp path] (1.504,3.279)--(1.684,3.279);
\draw[gp path] (7.829,3.279)--(7.649,3.279);
\node[gp node right] at (1.320,3.279) { 2.2};
\draw[gp path] (1.504,0.985)--(1.504,1.165);
\draw[gp path] (1.504,3.279)--(1.504,3.099);
\node[gp node center] at (1.504,0.677) {4};
\draw[gp path] (3.085,0.985)--(3.085,1.165);
\draw[gp path] (3.085,3.279)--(3.085,3.099);
\node[gp node center] at (3.085,0.677) {8};
\draw[gp path] (4.667,0.985)--(4.667,1.165);
\draw[gp path] (4.667,3.279)--(4.667,3.099);
\node[gp node center] at (4.667,0.677) {16};
\draw[gp path] (6.248,0.985)--(6.248,1.165);
\draw[gp path] (6.248,3.279)--(6.248,3.099);
\node[gp node center] at (6.248,0.677) {32};
\draw[gp path] (7.829,0.985)--(7.829,1.165);
\draw[gp path] (7.829,3.279)--(7.829,3.099);
\node[gp node center] at (7.829,0.677) {64};
\draw[gp path] (1.504,3.279)--(1.504,0.985)--(7.829,0.985)--(7.829,3.279)--cycle;
\node[gp node center,rotate=-270] at (0.246,2.132) {Relative performance};
\node[gp node center] at (4.666,0.215) {The number of columns};
\node[gp node right] at (3.382,4.238) {mul-tall-mem};
\gpcolor{color=gp lt color 0}
\gpsetlinetype{gp lt plot 0}
\draw[gp path] (3.566,4.238)--(4.482,4.238);
\draw[gp path] (1.504,2.059)--(3.085,1.788)--(4.667,1.913)--(6.248,1.913)--(7.829,1.673);
\gpsetpointsize{4.00}
\gppoint{gp mark 1}{(1.504,2.059)}
\gppoint{gp mark 1}{(3.085,1.788)}
\gppoint{gp mark 1}{(4.667,1.913)}
\gppoint{gp mark 1}{(6.248,1.913)}
\gppoint{gp mark 1}{(7.829,1.673)}
\gppoint{gp mark 1}{(4.024,4.238)}
\gpcolor{color=gp lt color border}
\node[gp node right] at (3.382,3.930) {mul-tall-EM};
\gpcolor{color=gp lt color 1}
\gpsetlinetype{gp lt plot 1}
\draw[gp path] (3.566,3.930)--(4.482,3.930);
\draw[gp path] (1.504,1.162)--(3.085,1.110)--(4.667,1.194)--(6.248,1.225)--(7.829,1.256);
\gppoint{gp mark 2}{(1.504,1.162)}
\gppoint{gp mark 2}{(3.085,1.110)}
\gppoint{gp mark 2}{(4.667,1.194)}
\gppoint{gp mark 2}{(6.248,1.225)}
\gppoint{gp mark 2}{(7.829,1.256)}
\gppoint{gp mark 2}{(4.024,3.930)}
\gpcolor{color=gp lt color border}
\node[gp node right] at (6.874,4.238) {mul-wide-mem};
\gpcolor{color=gp lt color 2}
\gpsetlinetype{gp lt plot 2}
\draw[gp path] (7.058,4.238)--(7.974,4.238);
\draw[gp path] (1.504,3.143)--(3.085,2.956)--(4.667,2.403)--(6.248,1.882)--(7.829,1.496);
\gppoint{gp mark 3}{(1.504,3.143)}
\gppoint{gp mark 3}{(3.085,2.956)}
\gppoint{gp mark 3}{(4.667,2.403)}
\gppoint{gp mark 3}{(6.248,1.882)}
\gppoint{gp mark 3}{(7.829,1.496)}
\gppoint{gp mark 3}{(7.516,4.238)}
\gpcolor{color=gp lt color border}
\node[gp node right] at (6.874,3.930) {mul-wide-EM};
\gpcolor{color=gp lt color 3}
\gpsetlinetype{gp lt plot 3}
\draw[gp path] (7.058,3.930)--(7.974,3.930);
\draw[gp path] (1.504,1.183)--(3.085,1.173)--(4.667,1.214)--(6.248,1.183)--(7.829,1.287);
\gppoint{gp mark 4}{(1.504,1.183)}
\gppoint{gp mark 4}{(3.085,1.173)}
\gppoint{gp mark 4}{(4.667,1.214)}
\gppoint{gp mark 4}{(6.248,1.183)}
\gppoint{gp mark 4}{(7.829,1.287)}
\gppoint{gp mark 4}{(7.516,3.930)}
\gpcolor{color=gp lt color border}
\gpsetlinetype{gp lt border}
\draw[gp path] (1.504,3.279)--(1.504,0.985)--(7.829,0.985)--(7.829,3.279)--cycle;
%% coordinates of the plot area
\gpdefrectangularnode{gp plot 1}{\pgfpoint{1.504cm}{0.985cm}}{\pgfpoint{7.829cm}{3.279cm}}
\end{tikzpicture}
%% gnuplot variables

		\vspace{-15pt}
		\caption{The performance of in-memory and EM implementations of two types
			of matrix multiplication: a tall matrix ($200M \times X$) times
			a small matrix ($X \times X$) and a wide matrix ($X \times 200M$)
			times a tall matrix ($200M \times X$). Their performance is
		normalized by the corresponding Intel MKL implementations.}
		\label{perf:mat_mul}
	\end{center}
\end{figure}

\dz{Further optimize the in-memory dense matrix multiplication.}

\subsection{Lazy evaluation}
How effective are the optimizations for a group of dense matrix operations?

In this section, we evaluate how aggressively we should materialize virtual
matrices. What is the sweet spot that balance computation and I/O.

We should plot a chart with multiple lines:
in-mem version with all matrices materialized,
in-mem version with matrices lazily evaluated,
EM version with all matrices materialized,
EM version with matrices lazily evaluated.

In this chart, we only count the time used by dense matrix operations.

\subsection{Compare EM eigensolvers with in-memory eigensolvers}
We compare our EM eigensolvers with our in-mem eigensolvers as well as the original
Anasazi eigensolvers. For the in-mem eigensolver, it should also have two versions.
One version is to materialize all dense matrix operations and the other version is
to materialize virtual matrices in the same approach as the EM eigensolvers.

\subsubsection{Tuning the eigensolver for faster convergence}
We need to trade off memory/disk space with speed and the number of writes.
The larger column matrix we store, the fewer iterations are required. However,
it is unclear how a large column matrix affects the amount of data written disks.
If we implement dense matrix operations naively, a large column matrix helps to
reduce the amount of data written to disks; if we avoid materializing a dense
matrix every time we generate a dense matrix, a smaller column matrix potentially
reduce the amount of data written to disks.

\begin{table}
	\begin{center}
		\small
		\begin{tabular}{|c|c|c|c|c|}
			\hline
			eigensolver & block size & the number of blocks \\
			\hline
			Krylov-Schur & 4 & \# eigenpairs \\
			\hline
			Davidson & \# eigenpairs & 8 \\
			\hline
			LOBPCG &  & N/A \\
			\hline
		\end{tabular}
		\normalsize
	\end{center}
	\caption{}
	\label{eigen_conf}
\end{table}

The block eigensolvers have two parameters that significantly affect
the convergence rate. One is the block size, which determines the number
of vectors in the subspace that operate together. The other is the number
of blocks. Table \ref{eigen_conf} summaries the configuration that performs
well in general for different numbers of eigenpairs and different eigenvalue
problems. The LOBPCG eigensolver is the special case that only allows users
to specify the block size.

The computation complexity increases roughly linearly with the number of blocks
and the square of the block size.

\subsubsection{Performance of eigensolvers}

When computing hundreds of eigenpairs, an EM eigensolver may be able to outperform
an in-mem version.

To compute hundreds of eigenpairs, we need a large block size, which means
the dense matrix has a large number of columns. EM multiplication of such
a large matrix has performance similar to in-mem version. Furthermore,
to compute such a large number of eigenpairs, we need a lot of memory to
store the vectors in the basis. The EM eigensolver can store more vectors
for the basis and thus has a faster convergance rate.

We need to show how much read and write we save with lazy evaluation. We should
also show how much multiplication the lazy evaluation causes.
