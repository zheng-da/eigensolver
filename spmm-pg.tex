\begin{tikzpicture}[gnuplot]
%% generated with GNUPLOT 4.6p4 (Lua 5.1; terminal rev. 99, script rev. 100)
%% Wed 22 Jul 2015 12:34:54 AM EDT
\path (0.000,0.000) rectangle (8.382,4.572);
\gpcolor{color=gp lt color border}
\gpsetlinetype{gp lt border}
\gpsetlinewidth{1.00}
\draw[gp path] (1.320,0.985)--(1.500,0.985);
\draw[gp path] (7.829,0.985)--(7.649,0.985);
\node[gp node right] at (1.136,0.985) { 0};
\draw[gp path] (1.320,1.505)--(1.500,1.505);
\draw[gp path] (7.829,1.505)--(7.649,1.505);
\node[gp node right] at (1.136,1.505) { 10};
\draw[gp path] (1.320,2.026)--(1.500,2.026);
\draw[gp path] (7.829,2.026)--(7.649,2.026);
\node[gp node right] at (1.136,2.026) { 20};
\draw[gp path] (1.320,2.546)--(1.500,2.546);
\draw[gp path] (7.829,2.546)--(7.649,2.546);
\node[gp node right] at (1.136,2.546) { 30};
\draw[gp path] (1.320,3.067)--(1.500,3.067);
\draw[gp path] (7.829,3.067)--(7.649,3.067);
\node[gp node right] at (1.136,3.067) { 40};
\draw[gp path] (1.320,3.587)--(1.500,3.587);
\draw[gp path] (7.829,3.587)--(7.649,3.587);
\node[gp node right] at (1.136,3.587) { 50};
\draw[gp path] (1.320,0.985)--(1.320,1.165);
\draw[gp path] (1.320,3.587)--(1.320,3.407);
\node[gp node center] at (1.320,0.677) {1};
\draw[gp path] (2.947,0.985)--(2.947,1.165);
\draw[gp path] (2.947,3.587)--(2.947,3.407);
\node[gp node center] at (2.947,0.677) {2};
\draw[gp path] (4.575,0.985)--(4.575,1.165);
\draw[gp path] (4.575,3.587)--(4.575,3.407);
\node[gp node center] at (4.575,0.677) {4};
\draw[gp path] (6.202,0.985)--(6.202,1.165);
\draw[gp path] (6.202,3.587)--(6.202,3.407);
\node[gp node center] at (6.202,0.677) {8};
\draw[gp path] (7.829,0.985)--(7.829,1.165);
\draw[gp path] (7.829,3.587)--(7.829,3.407);
\node[gp node center] at (7.829,0.677) {16};
\draw[gp path] (1.320,3.587)--(1.320,0.985)--(7.829,0.985)--(7.829,3.587)--cycle;
\node[gp node center,rotate=-270] at (0.246,2.286) {seconds};
\node[gp node center] at (4.574,0.215) {The number of columns of the dense matrix};
\node[gp node right] at (3.290,4.238) {SEM};
\gpcolor{color=gp lt color 0}
\gpsetlinetype{gp lt plot 0}
\draw[gp path] (3.474,4.238)--(4.390,4.238);
\draw[gp path] (1.320,3.587)--(2.947,3.525)--(4.575,3.509)--(6.202,3.431)--(7.829,0.985);
\gpsetpointsize{4.00}
\gppoint{gp mark 1}{(1.320,3.587)}
\gppoint{gp mark 1}{(2.947,3.525)}
\gppoint{gp mark 1}{(4.575,3.509)}
\gppoint{gp mark 1}{(6.202,3.431)}
\gppoint{gp mark 1}{(7.829,0.985)}
\gppoint{gp mark 1}{(3.932,4.238)}
\gpcolor{color=gp lt color border}
\node[gp node right] at (5.126,4.238) {mem};
\gpcolor{color=gp lt color 1}
\gpsetlinetype{gp lt plot 1}
\draw[gp path] (5.310,4.238)--(6.226,4.238);
\draw[gp path] (1.320,1.693)--(2.947,2.406)--(4.575,2.858)--(6.202,3.394)--(7.829,0.985);
\gppoint{gp mark 2}{(1.320,1.693)}
\gppoint{gp mark 2}{(2.947,2.406)}
\gppoint{gp mark 2}{(4.575,2.858)}
\gppoint{gp mark 2}{(6.202,3.394)}
\gppoint{gp mark 2}{(7.829,0.985)}
\gppoint{gp mark 2}{(5.768,4.238)}
\gpcolor{color=gp lt color border}
\gpsetlinetype{gp lt border}
\draw[gp path] (1.320,3.587)--(1.320,0.985)--(7.829,0.985)--(7.829,3.587)--cycle;
%% coordinates of the plot area
\gpdefrectangularnode{gp plot 1}{\pgfpoint{1.320cm}{0.985cm}}{\pgfpoint{7.829cm}{3.587cm}}
\end{tikzpicture}
%% gnuplot variables
